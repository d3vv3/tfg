\chapter{Conclusiones y futuras líneas de trabajo}
\label{chap:conclusions}

En este capítulo se exponen las conclusiones extraídas del proyecto, así como recomendaciones y posibles futuras líneas de trabajo.

\section{Conclusiones}
\label{sec:conclusions}

Este proyecto ha desarrollado una doble función: personalmente, el desarrollo de un primer proyecto para el mundo real, con casos de uso reales; académicamente me ha permitido profundizar en las técnicas y metodologías de desarrollo de software, concluyendo mi trabajo de fin de grado.

La motivación principal del proyecto era desarrollar una aplicación web para ayudar a analizar conversaciones de WhatsApp y Telegram, permitiendo detectar problemas y tomar acciones para mejorar.

En un primer paso, definimos varios casos de uso para poder definir correctamente nuestro sistema, así como recogimos los requisitos no funcionales del mismo. Con todo lo anterior definido, comenzamos el diseño de la arquitectura del sistema, donde presentamos a grandes rasgos los módulos que la aplicación debía tener. Salvando detalles de implementación, que pueden observarse en el \autoref{chap:code}, hemos encontrado dificultades técnicas como el aprendizaje de Rust para el desarrollo de módulos en \acrfull{wasm}, así como la integración de \acrshort{wasm} con el código en JavaScript. Durante esta fase de desarrollo, hemos implementado buenas prácticas de ingeniería de software, como el control de versiones, la escritura de software modular y reutilizable, y tests. Finalmente, se ha expuesto el caso de uso del sistema completo, comprobando el cumplimiento de los requisitos funcionales y no funcionales.

Personalmente, como usuario de la aplicación, este proyecto ha servido para mejorar amistades y relaciones de pareja, así como ayudado a comprender mejor el comportamiento de mis grupos de amigos y seres queridos.

\begin{comment}
	
Academically, this Project served a double purpose: learning about statistics and machine learning and developing a method for improving the accuracy of an existing system.

The initial motivation for the Project was to develop better estimations for the bus \acrshort{eta}s, and this main objective has been satisfactorily accomplished. Precisely, both of the goals proposed in the introduction (characterizing the \acrshort{crtm} \acrshort{api} estimations and developing alternative estimators) have been fullfilled.

The first step was to review the existing bibliography and check the available data and data sources from the \acrshort{crtm}. Then, we addressed the first challenge: obtaining as much data about the buses mobility as possible without saturating the \acrshort{crtm} \acrshort{api} server. After several tests, the server behaviour was characterized and decisions for an optimal data gathering were taken. Furthermore, the Python package \textbf{crtm\_poll} was developed for automating the server polling process. As a conclusion, the optimization of the data acquisition procedures is crucial in this type of scenarios.

Once the data was available, a data analysis environment was set up for its processing. An algorithm for estimating the bus passing time from the \acrshort{api} provided \acrshort{eta}s was designed, which involved filtering the raw data obtained from the \acrshort{crtm} \acrshort{api}. Knowing the arrival times of past trips, the running time between stops could be obtained and a web visualization tool was developed for displaying them. The developed tools allowed to visually detect some fundamental behavior patterns in the mobility of the buses.

Next, alternative estimators for the running times between stops and for the bus \acrshort{eta} to a stop were designed. For that purpose different statistical tools and machine learning models were employed. Our experience indicates that although the available software for machine learning model design is very powerful, a clear understanding of the addressed problem and of the training schemes is required to be able to correctly interpret the results.

Concerning the developed tools, the running time between stops estimators developed in Section \ref{running_time_between_two_stops} could be used to improve the accuracy of the estimations made by the \acrshort{crtm} \acrshort{api} or by other source of estimators; in general, they can also be applied to any scenario where we want to estimate the running time between two points of a vehicle following a fixed route.

The relevant idea behind the estimator developed in Section \ref{reamining_time_at_anytime} is that it can be understood as a system that corrects the estimations provided by another system. This could be an easy to implement solution for improving the accuracy of the estimations made by more complex systems just by setting it prior to them.

The main limitation for implementing the proposed online estimator is the bottleneck produced by the \acrshort{crtm} \acrshort{api} server, which would not be able to handle all the needed requests for sampling all the bus stops at the same time. Nevertheless, the computational cost of each of the used models was calculated in case that the processing capacity requirements were to be consider for a future implementation.

In general, the developed methods in this Project are built in a way that allows the usability for other similar use cases. The hyperparameters of the estimators are chosen based on the data and the available input features, so they can be easily adapted for different system behaviors.
\end{comment}

\section{Objetivos conseguidos}
\label{sec:achieved-goals}

Finalmente, hemos logrado alcanzar todos los objetivos propuestos en el \autoref{chap:introduction}. Se exponen a continuación en mayor detalle:

\paragraph{Importar datos conversacionales de aplicaciónes de mensajería instantánea.} Se ha desarrollado un módulo que permite importar datos conversacionales de distintas aplicaciones de mensajería instantánea; particularmente WhatsApp y Telegram, asentando las bases para la inclusión de más aplicaciones dentro del mismo módulo. Además, se pueden importar datos conversacionales de chats grupales e individuales en ambas aplicaciones.

\paragraph{Selección y cálculo de estadísticas de conversaciones.} Se han seleccionado e implementado el cálculo de métricas tales como el número de mensajes enviados por cada contacto, número de palabras y caracterers, así como media de palabras y caracteres por mensaje. También se han incluido métricas del tiempo medio de respuesta de cada contacto y número de conversaciones que ha iniciado; así como distribuciones de los mensajes por meses, días de la semana y horas del día, que permiten ver al usuario la evolución de la conversación en el tiempo y el comportamiento semanal y diario de sus chats. Finalmente se incluyen métricas de las palabras y los emoticonos más utilizados.

\paragraph{Diseño y desarrollo de las visualizaciones.} Se han diseñado y desarrollado visualizaciones interactivas para mostrar las métricas mencionadas en el objetivo anterior mediante un diseño reutilizable. Tales visualizaciones incluyen gráficos circulares, gráficos de barras, nubes de palabras y frases con las estadísticas resaltadas.

\paragraph{Integración multiplataforma.} Hemos considerado una integración absoluta con todos los dispositivos con navegador web, permitiendo el uso en dispositivos móviles, tabletas y ordenadores; permitiendo la instalación como \acrshort{pwa} en los navegadores compatibles. Con esto ofrecemos una sensación de mayor integración con el dispositivo, así como la capacidad de funcionamiento sin conexión a Internet. Todo esto es posible gracias a un diseño responsivo.

\paragraph{Aplicación de técnicas de ingeniería de software.} Durante todo el proyecto se han escrito numerosos tests para comprobar el correcto funcionamiento del código, así como se ha usado un sistema de control de versiones durante la fase de desarrollo. También ha constituido la consulta exhaustiva de la documentación de numerosas librerías de JavaScript, tales como React.



\section{Futuras líneas de trabajo}
\label{sec:future-work}

Con vistas a futuro, hemos asentado una base sólida sobre la que construir y escalar este proyecto. Es por ello que, para futuras contribuciones, se plantean las siguientes mejoras que han quedado fuera de cota para este trabajo:

\begin{itemize}

\item Migración de los módulos con mayor carga de trabajo a \acrfull{wasm}, permitiendo obtener un rendimiento mayor, cercano al nativo del cliente que ejecuta la aplicación.

\item Mayor facilidad para añadir módulos, mediante un sistema de carpetas y plugins.

\item Inclusión de visualizaciones para el análisis de sentimientos y su evolución en el tiempo.

\item Detección de grupos de interacción: qué contactos suelen responder entre ellos.

\item Nuevas métricas: contador de mensajes eliminados por cada contacto.

\item Mejora de la documentación para futuros contribuyentes al proyecto.

\item Aumento de la cobertura de tests unitarios, así como implementación de tests de extremo a extremo (\textit{end-to-end}) para cada caso de uso.

\end{itemize}
