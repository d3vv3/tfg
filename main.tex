\synctex=1
\documentclass[a4paper,11pt,svgnames]{book}

\usepackage[utf8x]{inputenc}
\usepackage{mathtools}
\usepackage{thesis}
\usepackage{calc}
\usepackage{float}
\usepackage{acronym}
\usepackage{siunitx}
\usepackage{eurosym}
\usepackage[titletoc]{appendix}

% \setlength{\marginparwidth}{2cm}
% \setlength{\voffset}{-0.25in}
\setlength{\parskip}{0.1in}

% \usepackage{draftwatermark}
% \SetWatermarkScale{4}
\usepackage{subfig}
\usepackage{listings}
\usepackage{courier}
\usepackage{enumitem}
% \usepackage{indentfirst}

\lstset{
    basicstyle=\footnotesize\ttfamily,
    numbers=none,               % Ort der Zeilennummern
    numberstyle=\tiny,          % Stil der Zeilennummern
    % stepnumber=2,               % Abstand zwischen den Zeilennummern
    numbersep=5pt,              % Abstand der Nummern zum Text
    tabsize=2,                  % Groesse von Tabs
    extendedchars=true,         %
    breaklines=true,            % Zeilen werden Umgebrochen
    % keywordstyle=\color{red},
    frame=single,         
    % keywordstyle=[1]\textbf,    % Stil der Keywords
    % keywordstyle=[2]\textbf,    %
    % keywordstyle=[3]\textbf,    %
    % keywordstyle=[4]\textbf,   \sqrt{\sqrt{}} %
    % stringstyle=\color{white}\ttfamily, % Farbe der String
    showspaces=false,           % Leerzeichen anzeigen ?
    showtabs=false,             % Tabs anzeigen ?
    belowcaptionskip=12pt,
    xleftmargin=2em, 
    xrightmargin=4pt,
    % backgroundcolor=\color{lightgray},
    showstringspaces=false,      % Leerzeichen in Strings anzeigen ?  
    aboveskip=5pt,
    belowskip=10pt
 }
 
 \lstloadlanguages{% Check Dokumentation for further languages ...
    % [Visual]Basic
    % Pascal
    % C
    % C++
    % XML
    % HP
    Java
 }
 
% \usepackage{xcolor}
\usepackage[usenames,dvipsnames,table]{xcolor}

\colorlet{punct}{red!60!black}
\definecolor{background}{HTML}{FFFFFF}
\definecolor{delim}{RGB}{20,105,176}
\colorlet{numb}{magenta!60!black}

\lstdefinelanguage{json}{
    basicstyle=\scriptsize\ttfamily,
    numbers=none,
    numberstyle=\scriptsize,
    stepnumber=1,
    numbersep=4pt,
    showstringspaces=false,
    breaklines=true,
    frame=lrtb,
    backgroundcolor=\color{background},
    literate=
     *{0}{{{\color{numb}0}}}{1}
      {1}{{{\color{numb}1}}}{1}
      {2}{{{\color{numb}2}}}{1}
      {3}{{{\color{numb}3}}}{1}
      {4}{{{\color{numb}4}}}{1}
      {5}{{{\color{numb}5}}}{1}
      {6}{{{\color{numb}6}}}{1}
      {7}{{{\color{numb}7}}}{1}
      {8}{{{\color{numb}8}}}{1}
      {9}{{{\color{numb}9}}}{1}
      {:}{{{\color{punct}{:}}}}{1}
      {,}{{{\color{punct}{,}}}}{1}
      {\{}{{{\color{delim}{\{}}}}{1}
      {\}}{{{\color{delim}{\}}}}}{1}
      {[}{{{\color{delim}{[}}}}{1}
      {]}{{{\color{delim}{]}}}}{1},
}
% \DeclareCaptionFont{blue}{\color{blue}} 
% \captionsetup[lstlisting]{singlelinecheck=false, labelfont={blue}, textfont={blue}}

\usepackage{caption}

% \DeclareCaptionFont{white}{\color{white}}
% \DeclareCaptionFormat{listing}{\colorbox[HTML]{6DBAFF}{\parbox{\textwidth}{\hspace{5pt}#1#2#3}}}

\DeclareCaptionFont{white}{\color{white}}
\captionsetup[lstlisting]{singlelinecheck=false, margin=0pt, box=colorbox, boxcolor=gray, font={color=white, bf, footnotesize}}

% \DeclareCaptionFormat{listing}{\colorbox{gray}{\parbox{\textwidth-20pt}{#1#2#3}}\vspace{0.01cm}}
% \captionsetup[lstlisting]{format=listing,labelfont=white,textfont=white}

\usepackage{url}
\usepackage{tabularx}
\usepackage{multirow}
\usepackage{graphicx}
\usepackage{verbatim}
\usepackage[section]{placeins}
\usepackage{listings}
\usepackage[spanish]{babel}
% \usepackage[T1]{fontenc}
% \usepackage[usenames,dvipsnames,table]{xcolor}
% \bibliographystyle{unsrt}

% glossaries
\usepackage[acronym,toc]{glossaries}
\makeglossaries
\newacronym{gnu}{GNU}{GNU's not Unix}
\newacronym{dom}{DOM}{Document Object Model}
\newacronym{ssh}{SSH}{Secure Shell}
\newacronym{ssl}{SSL}{Secure Sockets Layer}
\newacronym{pwa}{PWA}{Progressive Web App}
\newacronym{uml}{UML}{Unified Modelling Language}
\newacronym{json}{JSON}{JavaScript Object Notation}
\newacronym{wasm}{WASM}{Web Assembly}
\newacronym{tls}{TLS}{Transport Layer Security}
\newacronym{gplv3}{GPLv3}{GNU General Public License Version 3}
\newacronym{https}{HTTPS}{Hypertext Transfer Protocol Secure}


% \newacronym{crtm}{CRTM}{Madrid’s  public  transport  commission (\emph{Consorcio Regional de Transportes de Madrid})}

% \newacronym{gtfs}{GTFS}{General Transit Feed Specification \cite{gtfs_reference}}

\newcommand{\authorname}{JAIME CONDE SEGOVIA}
\newcommand{\tfgtitle}{DESIGN AND DEVELOPMENT OF A WEB TOOL TO ANALYSE INSTANT MESSAGING CHATS}
\newcommand{\tfgtitlees}{DISEÑO Y DESARROLLO DE UNA HERRAMIENTA WEB PARA ANÁLISIS DE CHATS DE MENSAJERÍA INSTANTÁNEA}
\newcommand{\supervisor}{JUAN FERNANDO SÁNCHEZ RADA}
\newcommand{\fecha}{ENERO 2023}

\usepackage[pdftex,
    pdfauthor={\authorname},
    pdftitle={\tfgtitlees},
    pdfsubject={Bachelor Final Project},
    pdfkeywords={GSI},
    pdfproducer={PDFTex},
    colorlinks=true,linkcolor=black,citecolor=black,urlcolor=black,hypertexnames=false]{hyperref}

% \usepackage{longtable}
\setcounter{secnumdepth}{3}

\usepackage{framed}

% lstlisting
\usepackage{listings}

\lstdefinelanguage{JavaScript}{
    keywords={typeof, new, true, false, catch, function, return, null, catch, switch, var, if, in, while, do, else, case, break},
    keywordstyle=\color{blue}\bfseries,
    ndkeywords={class, export, boolean, throw, implements, import, this},
    ndkeywordstyle=\color{darkgray}\bfseries,
    identifierstyle=\color{black},
    sensitive=false,
    comment=[l]{//},
    morecomment=[s]{/*}{*/},
    commentstyle=\color{purple}\ttfamily,
    stringstyle=\color{red}\ttfamily,
    morestring=[b]',
    morestring=[b]"
}

\lstdefinelanguage{yaml}{
  keywords={true, false, null, y, n},
  keywordstyle=\color{darkgray}\bfseries,
  sensitive=false,
  comment=[l]{\#},
  morecomment=[s]{/*}{*/},
  commentstyle=\color{purple}\ttfamily,
  stringstyle=\color{red}\ttfamily,
  moredelim=[l][\color{orange}]{\&},
  moredelim=[l][\color{magenta}]{*},
  % moredelim=**[il][\color{black}\mdseries{:}\color{black}]{:},   % switch to value style at :
  morestring=[b]',
  morestring=[b]",
  literate =    {---}{{\llap{\color{cyan}\mdseries-{-}-}}}3
                {>}{{\textcolor{red}\textgreater}}1     
                {|}{{\textcolor{red}\textbar}}1 
                {\ -\ }{{\mdseries\ -\ }}3,
}

\definecolor{darkblue}{rgb}{0.0,0.0,0.6}

\lstdefinestyle{listXML}{
    language=XML, basicstyle=\ttfamily\diny, extendedchars=true,  belowcaptionskip=5pt,xleftmargin=0.4em, xrightmargin=0.3em, numbers=none, frame=single, breaklines=true, breakatwhitespace=true, breakindent=0pt, emph={}, emphstyle=\color{red}, basicstyle=\small\ttfamily, columns=fullflexible, showstringspaces=false, commentstyle=\color{gray}\upshape,
    morestring=[b]",
    morecomment=[s]{<?}{?>},
    morecomment=[s][\color{orange}]{<!--}{-->},
    keywordstyle=\color{cyan},
    stringstyle=\color{black},
    tagstyle=\color{darkblue},
    morekeywords={xmlns,version,type}
}

\lstdefinestyle{mono}{
    framesep=8px,
    extendedchars=true,
    basicstyle=\ttfamily,
    showstringspaces=false,
    showspaces=false,
    tabsize=2,
    breaklines=true,
    showtabs=false,
    xleftmargin=8pt,
    xrightmargin=8pt
}

\lstdefinestyle{commands}{
    framesep=8px,
    extendedchars=true,
    basicstyle=\ttfamily,
    showstringspaces=false,
    showspaces=false,
    tabsize=2,
    breaklines=true,
    showtabs=false,
    xleftmargin=8pt,
    xrightmargin=8pt
}

\lstdefinestyle{consola}{
    basicstyle=\scriptsize\ttfamily,
    backgroundcolor=\color{white},
    frame=lrtb,
    numbers=none,
    xleftmargin=4pt,
    xrightmargin=4pt
}

% Allows to change the color of chapter headers
\definecolor{chapterdetails}{HTML}{00a9e0}

% \usepackage[sf,bf]{titlesec}
% \titleformat{\chapter}[display]
%   {\normalfont\Large\sffamily\raggedleft}
%   {\vspace{5cm}\MakeUppercase{\chaptertitlename}%
%     \rlap{ \resizebox{!}{1.5cm}{\thechapter} \color{chapterdetails}\rule{5cm}{1.5cm}}}
%   {10pt}{\Huge}[{\color{chapterdetails}\titlerule[0.8mm] }]
% \titlespacing*{\chapter}{0pt}{30pt}{20pt}

\usepackage[sf,bf]{titlesec}
\titleformat{\chapter}[display]
  {\normalfont\Large\sffamily\raggedleft}
  {\vspace{5cm}\MakeUppercase{\chaptertitlename}%
    \rlap{ \resizebox{!}{1.5cm}{\thechapter} \color{chapterdetails}\rule{5cm}{1.5cm}}}
  {10pt}{\Huge}[{\color{chapterdetails}\titlerule[0.8mm] }]
\titlespacing*{\chapter}{0pt}{30pt}{20pt}
\titlespacing*{\section}{0pt}{20pt}{10pt}

% \titleformat{\section}{\large\sffamily\bfseries}{\thesection}{1em}{}

\newenvironment{chapterintro}
{% This is the begin code
\large\it
}
{% This is the end code
}

% Tick symbols
% \newcommand{\tickYes}{\checkmark}
% \newcommand{\tickNo}{\hspace{1pt}\ding{55}}
\usepackage{pifont} % http://ctan.org/pkg/pifont
\newcommand{\tickYes}{\ding{51}}
\newcommand{\tickNo}{\ding{55}}

% Fancy header
\usepackage{fancyhdr}

\usepackage{subfig}

% Fancy chapter cover style

% Fancy box
\usepackage{fancybox} 
\setlength{\fboxrule}{1pt} 
\setlength{\fboxsep}{10pt} 
\setlength{\shadowsize}{3pt}

% Sky color definition

% Portada
\usepackage{eso-pic,graphicx}
% \usepackage{tikz}
% \usepackage[top=0cm, bottom=0cm, outer=0cm, inner=0cm]{geometry}

% Comments
\usepackage{todonotes}
\usepackage{soul}
\usepackage{hyperref}

\begin{document}

\newcommand\litem[1]{\item{\bfseries #1 }}
\renewcommand{\arraystretch}{1.5} % Makes tables less crammed

\newcommand\headcell[1]{
  \multicolumn{1}{|c|}{\cellcolor{DodgerBlue}\bfseries\sffamily\textcolor{white}{#1}}
}

% Comments
\newcommand{\tb}{\textcolor{blue}}
\newcommand{\tr}{\textcolor{red}}
\newcommand{\tod}{\todo[color=bluep]}
\newcommand{\tody}{\todo[inline,color=yellow]}
\newcommand{\todi}{\todo[inline,color=bluep]}
\definecolor{bluep}{RGB}{3, 252, 244}

% Cuadros por tablas
% \renewcommand{\listtablename}{Tables Index}
% \renewcommand{\tablename}{Table} 
% \renewcommand{•}{•}*{\lstlistingname}{List of X}

% Acronyms Definition
\acrodef{gsi}[GSI]{Grupo de Sistemas Inteligentes}

\pagenumbering{gobble}
\include{0-preamble}
\pagenumbering{Roman}
\cleardoublepage
\phantomsection
\chapter*{Resumen}
\addcontentsline{toc}{chapter}{Resumen}

Las aplicaciones de mensajería instantánea son un componente principal de las comunicaciones interpersonales; más aún para personas que no se ven en el día a día. En octubre de 2020, WhatsApp reportaba enviar unos 200 miles de millones de mensajes al día \cite{whatsAppsPerDay} y, a día de hoy, cuenta con más de 2 mil millones de usuarios.\cite{whatsAppsUsers} Globalmente en segundo lugar en lo referente a descargas, Telegram cuenta con 700 millones de usuarios \cite{telegramSecondPlaceGlobally} que envían más de 15 mil millones de mensajes al día. \cite{telegramMessagesPerDay}

Según Robin Dunbar, la capacidad del ser humano para mantener relaciones activas es limitada.\cite{dunbarNumber} Conocer el estado objetivo de las relaciones personales con nuestros círculos puede ser de vital importancia para el mantenimiento, conservación y mejora de relaciones fuertes, duraderas y sanas.

El principal objetivo de este trabajo es brindar al usuario una herramienta que ofrece estos datos objetivos de una forma visual y fácilmente comprensible. Estos datos pueden ayudar a los usuarios a tomar acciones para mejorar sus interacciones. Hemos decidido llamar a la herramienta: \textit{ChatStats}.

Por otro lado, los usuarios cada vez son más conscientes del valor de sus datos y los posibles usos negativos de la recolección de los mismos. La privacidad es una preocupación principal en el ámbito de las conversaciones personales, por lo que ha sido ampliamente considerada durante el desarrollo de este proyecto. Para proteger los datos del usuario, toda la aplicación se envía al cliente, en el que se realizan todas las operaciones necesarias con las conversaciones de manera local. 

Además, \textit{ChatStats} es de código libre, permitiendo el acceso al código para su lectura y modificación por parte de usuarios y contribuyentes; bajo la licencia \acrfull{gplv3}.\cite{GPLv3} La aplicación se ha desarrollado haciendo uso de React, \textit{framework} abierto desarrollado principalmente por Facebook.

Con todo, este proyecto presenta el diseño e implementación de una aplicación web que permite al usuario analizar sus conversaciones de WhatsApp y Telegram, obteniendo información sobre sus interacciones sociales y relaciones. Todo esto asegurando la privacidad de los datos, que nunca abandonan el dispositivo, y bajo una licencia de código libre.

\begin{comment}

Instant messaging applications are a major component of interpersonal communications; even more so for people who do not see each other on a day-to-day basis. In October 2020, WhatsApp reported sending about 200 billion messages per day and, as of today, it has more than 2 billion users.

In 1992, Robin Dunbar approximated the number of people who can fully relate to each other in a system by 150, relating it to the size of the brain's neocortex and its processing capacity. Knowing the objective state of personal relationships with our circles can be of vital importance for the maintenance, preservation and improvement of strong, lasting and healthy relationships.

The main objective of this project is to provide the user with a tool that offers this objective data in a visual and easily understandable way, allowing its analysis and subsequent action accordingly.

The human being generates and collects more and more data in the digital era. With the growth of this trend, users are increasingly aware of the value of their data and the use that can be made of it in the wrong hands. Privacy is a major concern in the realm of personal conversations, and it has been considered a primary concern during the development of this application. To protect user data, the entire application is sent to the client, and all necessary operations are executed on the client's device. Additionally, the application is open-source, and the code is freely accessible for reading and modification by users and contributors under the \acrfull{gplv3} license.\cite{GPLv3} The application is built using React.

Overall, this project presents the design and implementation of a web application that allows users to analyze their WhatsApp chats and gain insights into their social interactions and relationships, all while ensuring data privacy and accessibility under an open-source license. It is called \textit{ChatStats}.

Debido a la importancia de la privacidad de los datos en el ámbito de las conversaciones personales, se ha tenido en cuenta durante el desarrollo del proyecto como principal preocupación para la arquitectura.

Por ello se ha desarrollado una aplicación web que se envía al cliente en su totalidad para que todas las operaciones necesarias se ejecuten en su dispositivo. Además, todo el código es de acceso libre para lectura y modificación de usuarios y contribuyentes, categorizando el proyecto como \textit{software} libre y gratuito bajo la licencia \acrfull{gplv3}. \cite{GPLv3}
 
\end{comment}


\vfill
\textbf{Palabras clave: mensajería instantánea, aplicación web, JavaScript, código libre, WhatsApp, privacidad, número de Dunbar.} 
\cleardoublepage
\phantomsection
\chapter*{Abstract}
\addcontentsline{toc}{chapter}{Abstract}

Instant messaging applications are a major component of interpersonal communications; even more so for people who do not see each other on a day-to-day basis. In October 2020, WhatsApp reported sending about 200 billion messages per day \cite{whatsAppsPerDay} and, as of today, it has more than 2 billion users.\cite{whatsAppsUsers}

In 1992, Robin Dunbar approximated the number of people who can fully relate to each other in a system by 150, relating it to the size of the brain's neocortex and its processing capacity. \cite{dunbarNumber} Knowing the objective state of personal relationships with our circles can be of vital importance for the maintenance, preservation and improvement of strong, lasting and healthy relationships.

The main objective of this project is to provide the user with a tool that offers this unbiased data in a visual and easily understandable way, allowing its analysis and subsequent action accordingly.

% easily understandable way o easy to understand way (?)

The human being generates and collects more and more data in the digital era. With the growth of this trend, users are increasingly aware of the value of their data and the use that can be made of it in the wrong hands. Privacy is a major concern in the realm of personal conversations, and it has been considered a primary concern during the development of this application. To protect user data, the entire application is sent to the client, and all necessary operations are executed on the client's device. Additionally, the application is open-source, and the code is freely accessible for reading and modification by users and contributors under the \acrfull{gplv3} license.\cite{GPLv3} The application is built using React, a framework developed and maintained primarily by Facebook.

Overall, this project presents the design and implementation of a web application that allows users to analyze their WhatsApp chats and gain insights into their social interactions and relationships, all while ensuring data privacy and accessibility under an open-source license. It is called \textit{ChatStats}.

\begin{comment}
	The use of instant messaging applications has become a vital component of interpersonal communication, especially for individuals who do not have the opportunity to interact on a daily basis. WhatsApp, one of the most popular instant messaging platforms, reported sending approximately 200 billion messages per day in October 2020, and currently has over 2 billion users.
	
	In 1992, Robin Dunbar proposed Dunbar's number, which approximates the number of people with whom an individual can maintain stable social relationships at around 150. Understanding the objective state of personal relationships within our social circles can be crucial for maintaining, preserving, and improving strong, lasting, and healthy relationships.
	
	This thesis presents the design and implementation of a web application that aims to provide users with a tool to analyze and understand their WhatsApp chats in a visual and easily comprehensible way. The application allows users to gain insights into their social interactions and relationships, taking into account Dunbar's number.
	
	Privacy is a major concern in the realm of personal conversations, and it has been considered a primary concern during the development of this application. To protect user data, the entire application is sent to the client, and all necessary operations are executed on the client's device. Additionally, the application is open-source, and the code is freely accessible for reading and modification by users and contributors under the \acrfull{gplv3} license.\cite{GPLv3} The application is built using React.
	
	Overall, this thesis presents the design and implementation of a web application that allows users to analyze their WhatsApp chats and gain insights into their social interactions and relationships, all while ensuring data privacy and accessibility under an open-source license.
\end{comment}

\vfill
\textbf{Keywords: instante messaging, web application, JavaScript, open-source software, WhatsApp, privacy, Dunbar's number.} 
\cleardoublepage
\phantomsection
\chapter*{Agradecimientos}
\addcontentsline{toc}{chapter}{Agradecimientos}

Me gustaría expresar un especial agradecimiento a mi tutor, Juan Fernando Sánchez Rada, que en las buenas y en las malas, ha prestado ayuda en todo lo posible, ofreciendo apoyo y soluciones cuando ha sido necesario; en el proyecto y en lo personal.

Agradezco a mis compañeros de la universidad, familia y amigos por acompañarme durante esta etapa de mi vida y arrojar luz allá donde hubo sombra.

A mi compañero y amigo Carlos García-Mauriño Dueñas por alojarme una instancia de ShareLatex en su servidor, por sus consejos, así como por el incondicional apoyo que he recibido por su parte durante todo el proyecto.

A mi compañero y amigo Guillermo García Grao, por apoyar y contribuir en los primeros pasos del desarrollo del código de este proyecto.

Por último, quiero expresar mi agradecimiento al grupo {\sl GSI} en la ETSIT-UPM, que ha hecho posible realizar este trabajo sobre uno de mis primeros proyectos personales.

\begin{flushright}
	``In God we trust. All others must bring data.'' - W. Edwards Deming
\end{flushright}
% \include{D-acknowledgement} % Not to include
\include{E-contents}
\newacronym{gnu}{GNU}{GNU's not Unix}
\newacronym{dom}{DOM}{Document Object Model}
\newacronym{ssh}{SSH}{Secure Shell}
\newacronym{ssl}{SSL}{Secure Sockets Layer}
\newacronym{pwa}{PWA}{Progressive Web App}
\newacronym{uml}{UML}{Unified Modelling Language}
\newacronym{json}{JSON}{JavaScript Object Notation}
\newacronym{wasm}{WASM}{Web Assembly}
\newacronym{tls}{TLS}{Transport Layer Security}
\newacronym{gplv3}{GPLv3}{GNU General Public License Version 3}
\newacronym{https}{HTTPS}{Hypertext Transfer Protocol Secure}


% \newacronym{crtm}{CRTM}{Madrid’s  public  transport  commission (\emph{Consorcio Regional de Transportes de Madrid})}

% \newacronym{gtfs}{GTFS}{General Transit Feed Specification \cite{gtfs_reference}}

% Header style
\pagestyle{fancy}
\fancyhf{}
\fancyhead[RO]{\sffamily \slshape \rightmark}
\fancyhead[LE]{\sffamily \slshape \leftmark}
% \renewcommand{\footrulewidth}{0.4pt} % grosor de la línea del pie
\fancyfoot[OR,EL]{\rmfamily \thepage} % texto derecha del pie

\pagenumbering{arabic}
\chapter{Introducción}
\label{chap:introduction}

\section{Contexto y motivación}
\label{sec:context}

Con 14 años compré mi primer teléfono inteligente. Con ello, gran parte de las interacciones con las personas de mi entorno más cercano, tenían lugar a través de aplicaciones de mensajería instantánea: principalmente WhatsApp y Telegram. Por entonces no era consciente de la información que se perdía por estas vías. Con el tiempo me he dado cuenta de la importancia de la información no verbal que se manifiesta en gestos, emociones, entonación y; finalmente, con la acumulación de todos estos factores, puede llegar a deteriorar una relación si no se reacciona correctamente.

Han sido numerosas las ocasiones en las que he sentido la necesidad de comprobar si mi relación con alguna persona cercana estaba decayendo, o se trataba únicamente de un sentimiento sin fundamentos. Es por ello que quise enfocar este trabajo al desarrollo de una herramienta de código libre para el análisis de conversaciones exportadas por aplicaciones de mensajería instantánea. Al tratar información tan personal, la arquitectura estaba clara: todo el procesamiento debe hacerse en el cliente y evitar enviar información a servidores salvo la autorización del usuario final.

\section{Objetivos}
\label{sec:project-goals}


Con este trabajo se pretende diseñar una aplicación web que analice datos de conversaciones de WhatsApp y Telegram; y ofrezca al usuario una visualización de datos relevantes de la misma, tanto generales como de la evolución en el tiempo.

Podemos segregar los objetivos en los siguientes:

\begin{itemize}

\item Importar datos conversacionales de aplicaciones de mensajería instantánea.
\item Selección y cálculo de estadísticas de conversaciones.
\item Diseño y desarrollo de las visualizaciones.
\item Integración multiplataforma.
\item Aplicación de técnicas de ingeniería de software.

\end{itemize}

\section{Estructura del documento}
\label{sec:structure-of-document}

En esta sección describiremos la estructura de este documento, así como una breve descripción de los capítulos. La estructura es la siguiente:

\paragraph{Capítulo 1. Introducción.}

Se introduce la motivación que ha llevado al desarrollo del proyecto, así como los objetivos que busca alcanzar y resolver. Todo esto son ingredientes para la adecuada resolución de un problema. Asimismo, se busca explicar la estructura del documento.

\paragraph{Capítulo 2. Tecnologías habilitantes.}

En este capítulo se explicarán las tecnologías utilizadas que permiten la realización del proyecto, fundamentando la elección de las mismas de manera segregada en cliente, servidor y desarrollo.

\paragraph{Capítulo 3. Análisis de requisitos.}

Se detalla el análisis de requisitos funcionales en casos de uso, así como se recogen los requisitos no funcionales, reglas de dominio y restricciones.

\paragraph{Capítulo 4. Arquitectura.}

Se introducirá la arquitectura general, entrando en detalle en decisiones de diseño, módulos y unidades lógicas en las que se divide, así como la  implementación de las mismas.

\paragraph{Capítulo 5. Caso de estudio.}

Se expone el caso de estudio completo del sistema, mostrando los casos de uso en detalle. Se explica el funcionamiento completo desde el punto de vista del usuario.

\paragraph{Capítulo 6. Conclusiones y futuras líneas de trabajo.}

En este capítulo se extraen las conclusiones del proyecto, así como posibles formas de continuar el desarrollo del proyecto tras esta memoria.

\chapter{Tecnologías habilitantes}
\label{chap:enabling_technologies}

% Javascript, Reactjs, Docker, traefik, WebAssembly, WebApp, Regex, Git, Open Source Software, Linux

En este capítulo exploraremos las tecnologías habilitantes esenciales para el desarrollo e implementación de nuestra aplicación web. Estas tecnologías incluyen los lenguajes de programación, marcos, librerías y herramientas que han sido utilizaadas para construir la aplicación. Las tecnologías habilitadoras juegan un papel crucial en el desarrollo de cualquier aplicación de software, y es esencial elegir el conjunto correcto de tecnologías que se alineen con los objetivos y requisitos del proyecto.

\section{Tecnologías en el cliente}
\label{tec_hab:client}

Para el desarrollo del cliente, se han elegido las siguientes tecnologías:

\subsection{React}
\label{tec_hab:react}

% https://survey.stackoverflow.co/2022/#most-popular-technologies-webframe

Para el desarrollo del cliente web se ha elegido el framework React. React es una librería para el desarrollo de front-end en JavaScript. Es de código libre y permite construir interfaces de usuario mediante la definición de componentes reutilizables. Es impulsado y mantenido por Meta, así como una gran comunidad de individuos y compañías.

Cuenta con la mayor comunidad de desarrolladores frente a sus competidores: AngularJS, VueJS, NextJS. Esto facilita el desarrollo, con mayor cantidad de librerías disponibles y mayor facilidad para solventar errores debido a la cantidad de usuarios.

React y JSX forman un stack tecnológico que nos permite sustituir completamente el stack conformado por HTML, JavaScript y CSS, ganando modularidad durante el desarrollo.

Para modificar el contenido que se muestra al usuario, cuenta con acceso al \acrshort{dom}, que representa la página web como un árbol de nodos que pueden ser modificados con JavaScript.

\section{Progressive Web App}
\label{tec_hab:PWA}

El mercado de las aplicaciones está segmentado en función a las distintas tiendas de aplicaciones de cada plataforma. Esto hace que el desarrollo multiplataforma sea complejo. Recientemente han ido surgiendo frameworks como Flutter, aunque la comunidad de desarrollo es todavía pequeña.

Es por ello que para este proyecto nos hemos acogido a las \acrshort{pwa}, cuyo soporte ha sido desarrollado para navegadores basados en Chromium, así como Safari; mientras Firefox ha decidido no acoger el estándar de la Web abierta. \cite{firefoxNoPWA}

Mediante un archivo de manifiesto, se detalla el título, descripción, imágenes y acciones que la aplicación puede ejecutar o intervenir una vez instalada, permitiendo interacción con el sistema operativo.

Con ello, nos permite mantener una sola fuente de código mientras podemos ejecutar nuestra aplicación en la mayoría de los clientes web.

Además, en noviembre de 2016 el tráfico web móvil superó al tráfico web de escritorio. Desde entonces, se ha mantenido la distancia ligeramente, por lo que tendría sentido optimizar la aplicación hacia clientes móviles. \cite{movilTraficoMayor}

\subsection{WebAssembly}
\label{tec_hab:wam}


\acrfull{wasm} define un formato de código binario portable e instrucciones correspondientes a modo de interfaz para facilitar interacciones entre programas y el entorno del host. Se trata de un estándar abierto que apunta a soportar la ejecución de cualquier lenguaje en cualquier sistema operativo.

Para ello, requiere de la integración del soporte de \acrfull{wasm} por los navegadores. Los navegadores principales (Chrome, Firefox, Safari y Edge) ya soportan la versión 1.0 de WebAssembly.

WebAssembly nos permite ejecutar programas desde nuestra página web en el host con alto rendimiento, puesto que las instrucciones son precompiladas. Tarda menos en cargar al ser un binario y permite ejecutar órdenes a bajo nivel, obteniendo un rendimiento más cercano al nativo del sistema en el que se ejecuta (con menor número de capas intermediarias).

Hemos elegido Rust como lenguaje de desarrollo para los módulos \acrfull{wasm}, por su tendencia reciente de crecimiento, así como por su rendimiento en todo tipo de entornos.

\subsection{Expresiones regulares}
\label{tec_hab:regex}

Las expresiones regulares son una secuencia de caracteres y operadores especiales que definen y controlan una búsqueda de patrones y filtros en un texto de destino.

Haremos uso de las mismas para segmentar los mensajes recibidos en texto plano, obteniendo así diferentes grupos de texto que conformarán la fecha del mensaje, la hora, el nombre del contacto y el cuerpo del mensaje. Pese a ser conocidos por ser un problema más que una solución, con el correcto uso de la tecnología, puede simplificar mucho el desarrollo. Esto se debe a que podríamos obtener resultados similares mediante el uso de funciones que separen la cadena de caracteres en los lugares adecuados, eliminen caracteres innecesarios y formen los grupos anteriormente descritos, cosa que aumentaría la complejidad del desarrollo y añadiría unidades lógicas a la lógica de negocio.


\section{Tecnologías en el servidor}
\label{tec_hab:server}

Debido a la sencillez de nuestra lógica de negocio, no contamos con un \textit{back-end} en el servidor, dado que únicamente servimos la aplicación al cliente. Una vez servida la aplicación, todas las operaciones se ejecutan en el cliente. Con ello, se han seleccionado las siguientes tecnologías para el servidor:



\subsection{GNU/Linux}
\label{tec_hab:linux}

Pese a no haber dominado el mercado de los escritorios, Linux se encuentra en el $96.3\%$ de los servidores del mundo \cite{linuxMarketShare}. Se trata de un kernel de código libre, por lo que se alinea con este trabajo perfectamente. Junto con los programas asociados y necesarios para un servidor, como puede ser \acrshort{ssh}, nos permite operar y administrar el sistema para ejecutar programas y servicios a ofrecer.


\subsection{Traefik}
\label{tec_hab:traefik}

Un proxy inverso nos permite interceptar peticiones a nuestro servidor y redirigir el tráfico al servicio back-end adecuado. Además, permite la implementación de certificados \acrshort{ssl}, que facilita el tráfico seguro en las conexiones externas.

Hemos elegido Traefik frente a alternativas como Nginx o Apache, que, pese a ser contendientes más establecidos, no ofrecen una integración sencilla con Docker. Además, Traefik ofrece integración nativa con Let's Encrypt, autoridad de certificados sin ánimo de lucro que, mediante una prueba, expide certificados para los dominios bajo nuestra propiedad.


\section{Herramientas de desarrollo}
\label{tec_hab:project}

A lo largo del desarrollo del proyecto, hemos hecho uso de las siguientes tecnologías para facilitar el control y versionamiento del código, propiedad intelectual, despliegue de infraestructura y comprobación de errores.

\subsection{Git}
\label{tec_hab:git}

Git es un sistema de control de versiones distribuido, de código libre y gratuito, diseñado para manejar proyectos de cualquier tamaño, independientemente del número de contribuidores, de forma rápida y eficiente. Nos permite versionar cambios de nuestro código, así como recuperar versiones anteriores y desarrollar nuevas funciones en paralelo en diferentes ramas, entre otros.

Además, se trata de un sistema distribuido, por lo que evitamos la centralización del código; lo que nos permite mantener el control de versiones sin necesidad de estar conectados a Internet, ya que contamos con nuestro propio nodo local. Una vez tengamos acceso a Internet, podemos sincronizar nuestro trabajo con el nodo origen u otros.


\subsection{Docker}
\label{tec_hab:docker}

Para servir el cliente desarrollado en React, se ha utilizado Docker como tecnología de contenerización y virtualización ligera. Esto nos permitirá aislar nuestra aplicación en una capa superior al sistema operativo, permitiendo ejecutar la misma en cualquier sistema operativo basado en Linux, independientemente de las dependencias que este tenga instalado; siempre y cuando cuente con Docker.

Hemos optado por virtualización ligera para reducir el impacto en el servidor, permitiendo que este varíe sus recursos ocupados en función a la demanda dentro de las capacidades de nuestro servidor.


\subsection{Software Libre}
\label{tec_hab:foss}

Durante las fase de desarrollo, se usará Git como sistema de control de versiones, unido a la publicación total del código en repositorios de acceso libre y gratuito, pudiendo definirlo como ``Open Source Software'' bajo la licencia GNU General Public License v3.0 \cite{GPLv3}. Esta licencia nos permite, en resumen:

\begin{enumerate}
	\item Cualquiera puede copiar, modificar y distribuir este software.
	\item Se debe incluir la licencia con todas y cada distribución de este código.
	\item Se permite el uso privado de este software.
	\item Se permite el uso de este software con fines comerciales.
	\item En caso de construir un negocio únicamente de este código, se arriesga a publicar la fuente del resto del código derivado.
	\item En caso de modificación, se deben indicar los cambios realizados al código.
	\item Cualquier modificación de este código debe ser distribuida con la misma licencia, GPLv3.
	\item Este software se provee sin ningún tipo de garantía.
	\item Ni el autor del código ni la licencia pueden ser marcadas como responsables de daños producidos por el software.
\end{enumerate}

El código fuente está disponible en Codeberg para consulta, contribuciones y seguimiento de errores. Se ha elegido Codeberg como alternativa de código abierto a otras plataformas como GitHub, ya que está basada en Gitea. A su vez, existe un repositorio alternativo o \textit{mirror} en Github. Ambos pueden verse en el \autoref{chap:code}.


\subsection{Tests unitarios}
\label{tec_hab:tests}

A lo largo del desarrollo se han implementado numerosos tests unitarios para comprobar el funcionamiento de las diferentes unidades lógicas que componen la aplicación. Se ha utilizado Jest como librería para estas pruebas, puesto que es utilizada por Meta, que mantiene el marco de React y Jest, haciendo que ambas sean altamente compatibles y cuenten con gran cantidad de documentación.

\chapter{Análisis de requisitos}
\label{chap:use-case}

\section{Introducción}
\label{sec:introduction}

El siguiente diagrama UML representa un resumen visual de los casos de uso que se describirán a continuación: 

\begin{figure}[h]
	\centering
	\includegraphics[width=\textwidth]{img/uml.png}
	\caption{Diagrama \acrshort{uml}}
	\label{fig:chap3:uml}
\end{figure}

\section{Casos de uso}
\label{sec:use-cases}


\subsection{Consultar estadísticas y visualizaciones de chat individual sin contenido multimedia}

\subsubsection{Nombre del caso de uso} Consultar estadísticas y visualizaciones de chat individual sin contenido multimedia.

\subsubsection{Actores}

\begin{itemize}
	\item Usuario de WhatsApp.
	\item Navegador con soporte \acrfull{pwa}.
	\item Aplicación WhatsApp.
	\item Datos conversacionales de WhatsApp.
\end{itemize}


\subsubsection{Resumen} El usuario de WhatsApp usará un archivo previamente exportado desde la aplicación WhatsApp como archivo de entrada. Podrá hacerlo desde el explorador de archivos del navegador o compartiendo el archivo en formato \textit{txt} desde el menú de compartir de su sistema operativo, si tiene la \acrfull{pwa} instalada. El archivo no incluirá los mensajes multimedia ni su contenido. Una vez el archivo se ha introducido, el usuario puede confirmar la selección y comenzar el análisis del archivo de texto, cálculo de estadísticas y datos para gráficos. Tras el análisis, se mostrarán las diferentes estadísticas y visualizaciones en la ventana del navegador.

\subsubsection{Secuencia de acciones}

\begin{enumerate}
	\item El usuario introduce el archivo de entrada, ya sea mediante la selección desde el explorador de archivos del navegador, como compartiendo el archivo mediante la \acrshort{pwa} si estuviese instalada.
	\item El usuario confirma la selección del archivo de entrada.
	\item Se realiza el \textit{parseo} del texto, se calculan las estadísticas y se preparan las estructuras de datos para la visualización, mientras el usuario espera en una pantalla de carga. Estas operaciones se realizan en el cliente.
	\item Se muestran gráficos interactivos, así como las estadísticas calculadas previamente.
\end{enumerate}

\begin{comment}
	Debería crear un caso de uso para chats individuales y otro para chats grupales?
	Debería diferenciar el caso de uso de PWA y web normal? (Por la integración con el SO)
	El actor es el archivo exportado y no WhatsApp.
	He cancelado el caso de uso de Telegram puesto que solo puede exportarse desde el ordenador y solo para algunos clientes.
\end{comment}

\subsection{Actores del sistema}
\label{subsec:system-actors}

\subsubsection{Usuario de WhatsApp}

Se trata del usuario único en el que se centran nuestros casos de uso. Este es un usuario de WhatsApp, ya que ChatStats es compatible con los archivos de chat exportados por esta aplicación.

\subsubsection{Navegador con soporte \acrfull{pwa}}

Nuestra aplicación web se ejecuta en un navegador, y por tanto, se trata de un actor para nuestro sistema. Además, si el navegador soporta \acrfull{pwa}, la aplicación web puede ser instalada, ofreciendo mayor integración con el sistema operativo.

\subsubsection{Aplicación WhatsApp}

Para exportar y generar los datos conversacionales que ChatStats espera como entrada, es necesario hacer uso de la aplicación de WhatsApp. Desde dicha aplicación, se puede seleccionar si la exportación se debe realizar con archivos multimedia o sin los mismos.

\subsubsection{Datos conversacionales de WhatsApp}

Es el fichero de datos que nustro sistema espera como entrada. En este caso de uso, se trata de un fichero de texto plano con todos los datos conversacionales exportados por la aplicación WhatsApp para un chat individual, sin incluir los archivos multimedia.

\subsection{Especificación suplementaria}
\label{subsect:suplementary-specification}

\subsubsection{Reglas de dominio}

% Qué eran las reglas de dominio? Qué debería poner aquí?


\subsubsection{Requisitos no funcionales}

\begin{itemize}
	\item \textbf{Interfaces:} leer y analizar archivos de texto plano \textit{txt}. 
	
	\item \textbf{Operación:} Interfaz de Usuario para navegador web, smartphone y tablet.
	
	\item \textbf{Seguridad:} Los datos no deben ser enviados a ningún servidor externo sin la autorización previa del usuario. Estos datos, además, no deben ser almacenados de manera temporal o permanente en ningún servidor. Todas las conexiones entre cliente y servidor deben estar cifradas con SSL.
	
	\item \textbf{Portabilidad:} La aplicación podrá ejecutarse en los navegadores: Chrome, Firefox, Edge y Safari; siendo compatible con \acrshort{pwa} en los navegadores que lo soporten.
\end{itemize}

\subsubsection{Restricciones}
De implementación: lenguaje JavaScript tanto para el servidor como para el cliente, y Docker para el despliegue. El código ha de ser libre.

De interfaz: uso de archivos \textit{txt} para importar los chats exportados desde la aplicación WhatsApp. 

Legales: RGPD.










\subsection{Consultar estadísticas y visualizaciones de chat grupal sin contenido multimedia}

\subsubsection{Nombre del caso de uso} Consultar estadísticas y visualizaciones de chat grupal sin contenido multimedia.

\subsubsection{Actores}

Usuario de WhatsApp.
Navegador con soporte \acrfull{pwa}.
Aplicación WhatsApp
Datos conversacionales de WhatsApp.

\subsubsection{Resumen} El usuario de WhatsApp usará un archivo previamente exportado desde la aplicación WhatsApp como archivo de entrada. Podrá hacerlo desde el explorador de archivos del navegador o compartiendo el archivo en formato \textit{txt} desde el menú de compartir de su sistema operativo, si tiene la \acrfull{pwa} instalada. El archivo no incluirá los mensajes multimedia ni su contenido. Una vez el archivo se ha introducido, el usuario puede confirmar la selección y comenzar el análisis del archivo de texto, cálculo de estadísticas y datos para gráficos. Tras el análisis, se mostrarán las diferentes estadísticas y visualizaciones en la ventana del navegador.

\subsubsection{Secuencia de acciones}

\begin{enumerate}
	\item El usuario introduce el archivo de entrada, ya sea mediante la selección desde el explorador de archivos del navegador, como compartiendo el archivo mediante la \acrshort{pwa} si estuviese instalada.
	\item El usuario confirma la selección del archivo de entrada.
	\item Se realiza el \textit{parseo} del texto, se calculan las estadísticas y se preparan las estructuras de datos para la visualización, mientras el usuario espera en una pantalla de carga. Estas operaciones se realizan en el cliente.
	\item Se muestran gráficos interactivos, así como las estadísticas calculadas previamente.
\end{enumerate}

\begin{comment}
	Debería crear un caso de uso para chats individuales y otro para chats grupales?
	Debería diferenciar el caso de uso de PWA y web normal? (Por la integración con el SO)
	El actor es el archivo exportado y no WhatsApp.
	He cancelado el caso de uso de Telegram puesto que solo puede exportarse desde el ordenador y solo para algunos clientes.
\end{comment}

\subsection{Actores del sistema}
\label{subsec:system-actors}

\subsubsection{Usuario de WhatsApp}

Se trata del usuario único en el que se centran nuestros casos de uso. Este es un usuario de WhatsApp, ya que ChatStats es compatible con los archivos de chat exportados por esta aplicación.

\subsubsection{Navegador con soporte \acrfull{pwa}}

Nuestra aplicación web se ejecuta en un navegador, y por tanto, se trata de un actor para nuestro sistema. Además, si el navegador soporta \acrfull{pwa}, la aplicación web puede ser instalada, ofreciendo mayor integración con el sistema operativo.

\subsubsection{Aplicación WhatsApp}

Para exportar y generar los datos conversacionales que ChatStats espera como entrada, es necesario hacer uso de la aplicación de WhatsApp. Desde dicha aplicación, se puede exportal un chat grupal y seleccionar si la exportación se debe realizar con archivos multimedia o sin los mismos.

\subsubsection{Datos conversacionales de WhatsApp}

Es el fichero de datos que nustro sistema espera como entrada. En este caso de uso, se trata de un fichero de texto plano con todos los datos conversacionales exportados por la aplicación WhatsApp para un chat de grupo, sin incluir los archivos multimedia.

\subsection{Especificación suplementaria}
\label{subsect:suplementary-specification}

\subsubsection{Reglas de dominio}

% Qué eran las reglas de dominio? Qué debería poner aquí?


\subsubsection{Requisitos no funcionales}

\begin{itemize}
	\item \textbf{Interfaces:} leer y analizar archivos de texto plano \textit{txt}. 
	
	\item \textbf{Operación:} Interfaz de Usuario para navegador web, smartphone y tablet.
	
	\item \textbf{Seguridad:} Los datos no deben ser enviados a ningún servidor externo sin la autorización previa del usuario. Estos datos, además, no deben ser almacenados de manera temporal o permanente en ningún servidor. Todas las conexiones entre cliente y servidor deben estar cifradas con SSL.
	
	\item \textbf{Portabilidad:} La aplicación podrá ejecutarse en los navegadores: Chrome, Firefox, Edge y Safari; siendo compatible con \acrshort{pwa} en los navegadores que lo soporten.
\end{itemize}

\subsubsection{Restricciones}
De implementación: lenguaje JavaScript tanto para el servidor como para el cliente, y Docker para el despliegue. El código ha de ser libre.

De interfaz: uso de archivos \textit{txt} para importar los chats exportados desde la aplicación WhatsApp. 

Legales: RGPD.













\subsection{Consultar estadísticas y visualizaciones de chat individual con contenido multimedia}

\subsubsection{Nombre del caso de uso} Consultar estadísticas y visualizaciones de chat individual con contenido multimedia.

\subsubsection{Actores}

Usuario de WhatsApp.
Navegador con soporte \acrfull{pwa}.
Aplicación WhatsApp
Datos conversacionales de WhatsApp.

\subsubsection{Resumen} El usuario de WhatsApp usará un archivo previamente exportado desde la aplicación WhatsApp como archivo de entrada. Podrá hacerlo desde el explorador de archivos del navegador o compartiendo el archivo en formato \textit{txt} desde el menú de compartir de su sistema operativo, si tiene la \acrfull{pwa} instalada. El archivo incluirá los mensajes multimedia. Una vez el archivo se ha introducido, el usuario puede confirmar la selección y comenzar el análisis del archivo de texto, cálculo de estadísticas y datos para gráficos. Tras el análisis, se mostrarán las diferentes estadísticas y visualizaciones en la ventana del navegador, con las visualizaciones de contenido multimedia correspondientes.

\subsubsection{Secuencia de acciones}

\begin{enumerate}
	\item El usuario introduce el archivo de entrada, ya sea mediante la selección desde el explorador de archivos del navegador, como compartiendo el archivo mediante la \acrshort{pwa} si estuviese instalada.
	\item El usuario confirma la selección del archivo de entrada.
	\item Se realiza el \textit{parseo} del texto, se calculan las estadísticas y se preparan las estructuras de datos para la visualización, mientras el usuario espera en una pantalla de carga. Estas operaciones se realizan en el cliente.
	\item Se muestran gráficos interactivos, así como las estadísticas calculadas previamente.
\end{enumerate}

\begin{comment}
	Debería crear un caso de uso para chats individuales y otro para chats grupales?
	Debería diferenciar el caso de uso de PWA y web normal? (Por la integración con el SO)
	El actor es el archivo exportado y no WhatsApp.
	He cancelado el caso de uso de Telegram puesto que solo puede exportarse desde el ordenador y solo para algunos clientes.
\end{comment}

\subsection{Actores del sistema}
\label{subsec:system-actors}

\subsubsection{Usuario de WhatsApp}

Se trata del usuario único en el que se centran nuestros casos de uso. Este es un usuario de WhatsApp, ya que ChatStats es compatible con los archivos de chat exportados por esta aplicación.

\subsubsection{Navegador con soporte \acrfull{pwa}}

Nuestra aplicación web se ejecuta en un navegador, y por tanto, se trata de un actor para nuestro sistema. Además, si el navegador soporta \acrfull{pwa}, la aplicación web puede ser instalada, ofreciendo mayor integración con el sistema operativo.

\subsubsection{Aplicación WhatsApp}

Para exportar y generar los datos conversacionales que ChatStats espera como entrada, es necesario hacer uso de la aplicación de WhatsApp. Desde dicha aplicación, se puede seleccionar si la exportación se debe realizar con archivos multimedia o sin los mismos. Este caso de uso comprende la exportación con contenido multimedia.

\subsubsection{Datos conversacionales de WhatsApp}

Es el fichero de datos que nustro sistema espera como entrada. En este caso de uso, se trata de un fichero de texto plano con todos los datos conversacionales exportados por la aplicación WhatsApp para un chat individual, incluyendo los archivos multimedia.

\subsection{Especificación suplementaria}
\label{subsect:suplementary-specification}

\subsubsection{Reglas de dominio}

% Qué eran las reglas de dominio? Qué debería poner aquí?


\subsubsection{Requisitos no funcionales}

\begin{itemize}
	\item \textbf{Interfaces:} leer y analizar archivos de texto plano \textit{txt}. 
	
	\item \textbf{Operación:} Interfaz de Usuario para navegador web, smartphone y tablet.
	
	\item \textbf{Seguridad:} Los datos no deben ser enviados a ningún servidor externo sin la autorización previa del usuario. Estos datos, además, no deben ser almacenados de manera temporal o permanente en ningún servidor. Todas las conexiones entre cliente y servidor deben estar cifradas con SSL.
	
	\item \textbf{Portabilidad:} La aplicación podrá ejecutarse en los navegadores: Chrome, Firefox, Edge y Safari; siendo compatible con \acrshort{pwa} en los navegadores que lo soporten.
\end{itemize}

\subsubsection{Restricciones}
De implementación: lenguaje JavaScript tanto para el servidor como para el cliente, y Docker para el despliegue. El código ha de ser libre.

De interfaz: uso de archivos \textit{txt} para importar los chats exportados desde la aplicación WhatsApp. 

Legales: RGPD.







\subsection{Consultar estadísticas y visualizaciones de chat grupal con contenido multimedia}

\subsubsection{Nombre del caso de uso} Consultar estadísticas y visualizaciones de chat grupal con contenido multimedia.

\subsubsection{Actores}

Usuario de WhatsApp.
Navegador con soporte \acrfull{pwa}.
Aplicación WhatsApp
Datos conversacionales de WhatsApp.

\subsubsection{Resumen} El usuario de WhatsApp usará un archivo previamente exportado desde la aplicación WhatsApp como archivo de entrada. Podrá hacerlo desde el explorador de archivos del navegador o compartiendo el archivo en formato \textit{txt} desde el menú de compartir de su sistema operativo, si tiene la \acrfull{pwa} instalada. El archivo incluirá los mensajes multimedia del grupo. Una vez el archivo se ha introducido, el usuario puede confirmar la selección y comenzar el análisis del archivo de texto, cálculo de estadísticas y datos para gráficos. Tras el análisis, se mostrarán las diferentes estadísticas y visualizaciones en la ventana del navegador, con las visualizaciones de contenido multimedia correspondientes.

\subsubsection{Secuencia de acciones}

\begin{enumerate}
	\item El usuario introduce el archivo de entrada, ya sea mediante la selección desde el explorador de archivos del navegador, como compartiendo el archivo mediante la \acrshort{pwa} si estuviese instalada.
	\item El usuario confirma la selección del archivo de entrada.
	\item Se realiza el \textit{parseo} del texto, se calculan las estadísticas y se preparan las estructuras de datos para la visualización, mientras el usuario espera en una pantalla de carga. Estas operaciones se realizan en el cliente.
	\item Se muestran gráficos interactivos, incluyendo las visualizaciones multimedia, así como las estadísticas calculadas previamente.
\end{enumerate}

\subsection{Actores del sistema}
\label{subsec:system-actors}

\subsubsection{Usuario de WhatsApp}

Se trata del usuario único en el que se centran nuestros casos de uso. Este es un usuario de WhatsApp, ya que ChatStats es compatible con los archivos de chat exportados por esta aplicación.

\subsubsection{Navegador con soporte \acrfull{pwa}}

Nuestra aplicación web se ejecuta en un navegador, y por tanto, se trata de un actor para nuestro sistema. Además, si el navegador soporta \acrfull{pwa}, la aplicación web puede ser instalada, ofreciendo mayor integración con el sistema operativo.

\subsubsection{Aplicación WhatsApp}

Para exportar y generar los datos conversacionales que ChatStats espera como entrada, es necesario hacer uso de la aplicación de WhatsApp. Desde dicha aplicación, se puede seleccionar si la exportación se debe realizar con archivos multimedia o sin los mismos. Este caso de uso comprende la exportación con contenido multimedia.

\subsubsection{Datos conversacionales de WhatsApp}

Es el fichero de datos que nustro sistema espera como entrada. En este caso de uso, se trata de un fichero de texto plano con todos los datos conversacionales exportados por la aplicación WhatsApp para un chat grupal, incluyendo los archivos multimedia.

\subsection{Especificación suplementaria}
\label{subsect:suplementary-specification}

\subsubsection{Reglas de dominio}

% Qué eran las reglas de dominio? Qué debería poner aquí?


\subsubsection{Requisitos no funcionales}

\begin{itemize}
	\item \textbf{Interfaces:} leer y analizar archivos de texto plano \textit{txt}. 
	
	\item \textbf{Operación:} Interfaz de Usuario para navegador web, smartphone y tablet.
	
	\item \textbf{Seguridad:} Los datos no deben ser enviados a ningún servidor externo sin la autorización previa del usuario. Estos datos, además, no deben ser almacenados de manera temporal o permanente en ningún servidor. Todas las conexiones entre cliente y servidor deben estar cifradas con SSL.
	
	\item \textbf{Portabilidad:} La aplicación podrá ejecutarse en los navegadores: Chrome, Firefox, Edge y Safari; siendo compatible con \acrshort{pwa} en los navegadores que lo soporten.
\end{itemize}

\subsubsection{Restricciones}
De implementación: lenguaje JavaScript tanto para el servidor como para el cliente, y Docker para el despliegue. El código ha de ser libre.

De interfaz: uso de archivos \textit{txt} para importar los chats exportados desde la aplicación WhatsApp. 

Legales: RGPD.
\chapter{Arquitectura}
\label{chap:architecture}


\section{Introducción}
\label{sec:introduction}

En este capítulo trataremos la fase de diseño de este proyecto, así como los detalles de implementación de la arquitectura del mismo. Primero, presentaremos una vista general del proyecto por medio de un diagrama. A continuación se detallará la arquitectura en el cliente, así como la del servidor y los servicios que la componen. Posteriormente, se detallarán los módulos que conforman la aplicación, así como las métricas que se han decidido calcular y sus motivos. Con ello tenemos la intención de facilitar al lector una vista general de la arquitectura del proyecto.

Se presenta a continuación una vista general de la arquitectura de ChatStats:

\begin{figure}[H]
	\centering
	\includegraphics[width=\textwidth]{img/architecture.png}
	\caption{Arquitectura general}
	\label{fig:chap4:architecture_general}
\end{figure}


\begin{comment}
	Como se puede observar, el usuario debe exportar un chat desde la aplicación de WhatsApp (móvil) o Telegram para escritorio. Son estas las únicas versiones de las aplicaciones que permiten exportar las conversaciones. Con estos datos exportados, el usuario accede a ChatStats desde el navegador, ya sea en un dispositivo móvil o en un ordenador. El servidor le envía el cliente completo, por lo que no intercambia más peticiones en adelante. El usuario puede seleccionar e importar el archivo a la aplicación, que los estandariza, calcula métricas sobre la conversación y ofrece una visualización de los mismos. Finalmente, el usuario puede compartir las visualizaciones, así como exportar los datos calculados para su uso personal.
	
	
	Se desarrolla toda la aplicación en \textit{JavaScript}, puesto que puede ejecutarse en todos los navegadores salvo que lo tengan deshabilitado. Otros lenguajes como \textit{PHP} o Python requieren de un servidor para realizar operaciones y que estos envíen una plantilla rellena con los resultados.
\end{comment}


\section{Arquitectura en el cliente}

A continuación se presenta la arquitectura que podemos encontrar en un cliente cualquiera. Puede tomarse la \autoref{fig:chap4:architecture_client} para observar las capas que la componen.

\begin{figure}[h]
	\centering
	\includegraphics[width=0.6\textwidth]{img/client.png}
	\caption{Arquitectura en el cliente}
	\label{fig:chap4:architecture_client}
\end{figure}

\paragraph{Navegador.} El navegador juega un papel fundamental para el acceso y ejecución de cualquier aplicación web. Aunque no se entra en detalle en su estructura interna, sí que vamos a detallar la integración con \acrfull{pwa}. Esta integración que ofrecen algunos navegadores como los basados en Chromium, permite instalar aplicaciones web, con ventajas como:

\begin{itemize}
	\item La aplicación saldrá en el escritorio en teléfonos Android e iOS, así como en el cajón de aplicaciones del navegador.
	\item Posibilidad de acceso a notificaciones \textit{push} para el sistema (que no utilizaremos).
	\item Posibilidad de guardar en \textit{cache} el código del cliente, permitiendo su uso sin conexión a Internet.
\end{itemize}

Es por ello que ChatStats puede ser instalada en los dispositivos con un navegador con soporte \acrshort{pwa}, permitiendo su instalación y ejecución sin acceso a Internet. Para ello, ChatStats registra un \textit{service worker} o trabajador de servicio que guarda el código en el cliente, así como un archivo de manifiesto que recoge un título, descripción e iconos de la aplicación.

Hay que mencionar también que Mozilla no ofrece soporte para \acrshort{pwa} en su navegador Firefox\cite{firefoxNoPWA}, aunque este puede ser habilitado mediante una extensión\cite{firefoxPWAextension}.

\paragraph{Aplicación Web.} En esta última capa se encuentra nuestra aplicación, cuya arquitectura de procesamiento se explica en la \autoref{chap:architecture:processing}. Explicamos a continuación los componentes lógicos que se encuentran en el cliente:

\subparagraph{React State.} Se trata de la capa que maneja el estado de nuestra aplicación, en el que se almacenan los resultados de los cálculos, importación de archivos y otros, como se verá en la \autoref{chap:architecture:processing}. Permite además que nuestra aplicación reaccione a los cambios sucedidos en esta capa de estado, pudiendo refrescar componentes que hagan uso de la información almacenada si fuese necesario.

\subparagraph{Router.} ChatStats es una aplicación multipágina. Esto quiere decir que cuenta con diferentes rutas web en las que se muestran diferentes páginas. La página principal consolida la ruta `\textit{/}', mientras que `\textit{/graphs}' enruta la página para la visualización de los gráficos. Usamos el enrutador proporporcionado por React para permitir la navegación por la aplicación, mapeando las rutas a los componentes de React que se deben visualizar en cada una. Estos se explicarán en la \autoref{chap:architecture:processing}.

\subparagraph{React Virtual DOM.} El DOM virtual es un concepto de programación en el que una representación virtual de la interfaz de usuario (UI) es guardada en memoria y sincronizada con el DOM del navegador. Esto nos permite definir qué queremos en la interfaz de usuario y React conseguirá que el DOM virtual y el DOM del navegador se sincronicen.


Por último, cabe añadir que se han implementado reglas de estilo que tienen en cuenta el tamaño de la pantalla y se adaptan al mismo, permitiendo mantener la usabilidad del cliente en teléfonos, tabletas y ordenadores.





\section{Arquitectura en el servidor}
\label{chap:architecture:server}

Se muestra a continuación una grafo de la arquitectura en el servidor, donde se muestran las capas que lo componen.

\begin{figure}[H]
	\centering
	\includegraphics[width=0.6\textwidth]{img/server.png}
	\caption{Arquitectura en el servidor}
	\label{fig:chap4:architecture_server}
\end{figure}

Se ha decidido no instalar el software directamente sobre el sistema operativo, evitando problemas de dependencias y distintas versiones de las mismas para los componentes del sistema operativo. Asimismo, se evitan problemas de seguridad que puedan venir por vulnerabilidades en el código fuente y sus dependencias.

Hemos elegido virtualización ligera para ejecutar nuestro código en contenedores, por las razones que se exponen:

\begin{itemize}
	\item Se contienen las dependencias de terceros en una imagen.
	\item En caso de vulnerabilidad, solo se expone el contenedor y no el sistema completo.
	\item Los recursos se ocupan dinámicamente en función a las necesidades, al contrario que con la virtualización completa.
	\item Permite el despliegue en cualquier sistema operativo compatible con Linux, salvo arquitecturas ARM (que no es frecuente en servidores).
\end{itemize}

\paragraph{ChatStats React Server.} Se trata del servidor de React que sirve el contenido. Tras construir una imagen con la versión de producción a partir del código fuente, este contenedor sirve el contenido estático final, que enviará al cliente completamente cuando este solicite la aplicación web. Se ha implementado por medio de un fichero Dockerfile, que parte de una imagen de \textit{NodeJS}, instala las dependencias y sirve el contenido. Con esta secuencia de instrucciones, se ha creado una imagen para uso en contenedores Docker, que puede encontrarse públicamente en el repositorio de imágenes DockerHub.

\paragraph{Traefik} Se ha decidido usar Traefik como \textit{proxy} inverso, que se sitúa frente al servidor de ChatStats para redirigir las peticiones realizadas a su contenedor correspondiente en el puerto adecuado.

Además, Traefik gestiona los certificados \acrshort{ssl} haciendo uso de \textit{Let's Encrypt}: autoridad sin ánimo de lucro que provee certificados para la capa \acrshort{tls} sin coste alguno.








\section{Arquitectura de la aplicación}
\label{chap:architecture:processing}


Con el objetivo de mantener la privacidad de los datos del usuario, se ha planteado una arquitectura centrada en el cliente, donde el servidor únicamente envía la totalidad de la aplicación al cliente en la primera petición. Esto supone un mayor coste computacional en el cliente para realizar todas las operaciones necesarias, por lo que la eficiencia del código es necesaria. Además, esta arquitectura se puede extender, en un futuro, ofreciendo analíticas adicionales si el usuario opta por enviar información al servidor para su procesamiento. Este caso de extensión se detallará más adelante.

A continuación se muestra la arquitectura de la lógica de negocio en la aplicación:

\begin{figure}[H]
	\centering
	\includegraphics[width=\textwidth]{img/architecture_processing.png}
	\caption{Arquitectura de la aplicación}
	\label{fig:chap4:architecture_processing}
\end{figure}

\subsection{Ingesta de datos}
\label{chap:architecture_ingesta}

El módulo importador se encarga de leer el fichero de entrada exportado desde la aplicación WhatsApp o Telegram y convertirlos a una estructura de datos normalizada para el procesamiento de los módulos posteriores. Dicho fichero puede estar en tres formatos diferentes:

\begin{itemize}
	\item \textit{txt} si se trata de un chat exportado de WhatsApp en un dispositivo Android.
	\item \textit{zip} si se trata de un chat exportado de WhatsApp desde un dispositivo iOS.
	\item \textit{\acrshort{json}} si se trata de un chat exportado de Telegram desde la aplicación para escritorio.
\end{itemize}

Aunque Telegram también soporta la exportación de conversaciones en \textit{HTML}, ChatStats no soporta este formato, puesto que está diseñado para la visualización y no para el tratamiento y procesado del mismo.


Mientras que Telegram ofrece los datos de exportación en formato \acrshort{json}, facilitando así la operabilidad de los mismos; WhatsApp no ofrece una estructura de datos ni consistencia entre distintas versiones de su aplicación. Es por ello que, en busca de la reutilización del código y escalabilidad del mismo, este módulo tiene como salida la siguiente estructura de datos:

\begin{lstlisting}[language=JavaScript]
	{
		date: new Date("2022-10-17T10:37:00"),
		from: "Juan Pedro",
		text: "IMG-20221020-WA0013.jpg",
		type: "message",
		media_type: "image"
	}
\end{lstlisting}

Esta estructura de datos conforma el máximo común divisor de la información que podemos obtener de Telegram y WhatsApp, por lo que es un buen punto de partida.

Además, en una primera instancia, cabe destacar que ChatStats únicamente hace uso de los metadatos del contenido multimedia, tales como formato de archivo, extensión y fecha. Estos datos se incluyen en el fichero que la aplicación espera a la entrada. Esto se detallará más adelante en este mismo módulo.


\paragraph{Entrada}\mbox{}\\

Este submódulo se encarga de cargar el fichero seleccionado por el usuario, que el navegador ofrece desde una ruta virtual y protegida, para que la aplicación no pueda acceder a todos los archivos del dispositivo. En caso de tratarse de un fichero de texto plano, este módulo abre el archivo como una cadena de caracteres. En caso de tratarse de un fichero comprimido, se lo envía al módulo ``Descompresor'' esperando un archivo de texto plano como salida. Por último, en caso de tratarse de cualquier otro tipo de fichero, el módulo alerta al usuario de que el archivo de entrada no es válido.

Este submódulo hace uso de una instancia de \textit{FileReader}; clase que ofrecen los navegadores, permitiendo abrir el fichero como secuencia de caracteres.

\paragraph{Descompresor (Opcional)}\mbox{}\\

Este submódulo se encarga de la descompresión de ficheros \textit{zip}, que descomprime bajo petición del módulo anterior. Finalmente, el módulo devuelve un fichero llamado \textit{\_chat.txt}, que, como se ha comentado anteriormente, contiene toda la información necesaria para el análisis (con o sin multimedia).

El descompresor hace uso de la librería \textit{jszip} para la descompresión del fichero. La lógica añadida a esta aplicación nos permite encontrar el fichero de texto plano dentro del fichero comprimido y devolver el mismo al módulo de ``Procesamiento WhatsApp''.


\paragraph{Procesamiento Telegram}\mbox{}\\

El objetivo de este submódulo es transformar el modelo de datos que exporta Telegram a el modelo de datos que utiliza ChatStats para el resto de sus módulos. Un ejemplo del objeto \acrshort{json} exportado por Telegram puede observarse en el \autoref{chap:telegram_json}.

En este caso, el submódulo de ``Entrada'' nos envía una cadena de caracteres en formato \acrshort{json}. Haciendo uso de la librería \textit{JSON} nativa para \textit{JavaScript}, podemos parsear el contenido de la cadena para obtener el objeto \acrshort{json}.

Debido a que Telegram ofrece una estructura que puede utilizarse para chats individuales o grupales, además de ofrecer soporte para mensajes multimedia, ChatStats hereda ciertas claves de dicha estructura. Es por ello que este submódulo elimina todas las claves no necesarias del objeto mensaje original de Telegram, quedándonos únicamente con las siguientes claves:

\begin{itemize}
	\item \textbf{``from'':} almacena el nombre del contacto. Admite cualquier secuencia de caracteres.
	\item \textbf{``type'':} almacena el tipo de mensaje. Actualmente su valor es \textit{``message''}, pero se ha considerado útil para futura escalabilidad del proyecto, dando soporte a videollamadas y cambio de nombre de grupo (anuncios), entre otros.
	\item \textbf{``date'':} almacena la fecha y hora del mensaje.
	\item \textbf{``media\_type'':} almacena el tipo de contenido multimedia. Puede tomar como valor: \textit{``voice\_message''}, \textit{``video\_file''}, \textit{``sticker''} o \textit{``image''}
\end{itemize}

\paragraph{Procesamiento WhatsApp}\mbox{}\\

Este submódulo se encarga, al igual que el submódulo anterior, de obtener mensajes con la estructura descrita al principio del módulo, a partir del archivo exportado de WhatsApp.

A continuación se muestra un mensaje exportado desde WhatsApp por un dispositivo Android:

\begin{lstlisting}
	17/07/2022, 01:28 - Alice: Este es un mensaje de prueba.
\end{lstlisting}

Así como un mensaje exportado desde WhatsApp por un dispositivo iOS:

\begin{lstlisting}
	[29/12/22, 0:14:55] Bob: Te escribo desde mi iPhone.
\end{lstlisting}

Podemos observar que, en el caso de WhatsApp, ambos mensajes se componen de la fecha, la hora, el nombre del contacto y el propio cuerpo del mensaje, aunque con distinto formato.

\subparagraph{Parseador de mensajes} Se implementa, como puede verse en el \autoref{chap:regex} un parseador mediante una expresión regular. Dicho parseador encuentra los mensajes en el texto recibido del submódulo de ``Entrada'' y los divide en: fecha y hora, contacto y cuerpo del mensaje. Podemos observar que no encontramos ninguna referencia al contenido multimedia aún, puesto que para ello hay que parsear el cuerpo del mensaje en busca de metadatos. Además, la fecha y la hora son una cadena de caracteres, que transformaremos a continuación.

\subparagraph{Parseador de fecha y hora} Este componente tiene como entrada la fecha y la hora del mensaje obtenido en el paso anterior. Devuelve un objeto de tipo \textit{Date}, nativo de JavaScript. Esto nos permite realizar operaciones de tiempo con facilidad en otros módulos.

Este submódulo utiliza la librería \textit{momentjs} para parsear los posibles distintos formatos de fecha que utilizan Telegram y WhatsApp en sus cadenas de caracteres. Además, considera el \textit{locale} del cliente, invirtiendo mes y día para el caso \textit{en\_US}.

\subparagraph{Parser de archivos multimedia} Este componente es escribir el tipo de contenido en la clave \textit{``media\_type''} del objeto mensaje expuesto anteriormente, buscando la extensión del archivo adjunto.

Se indican a continuación mensajes de ejemplo para cada tipo de archivo adjunto, únicamente para Android:

\begin{lstlisting}
	17/10/2022, 21:11 - Juan Pedro: PTT-20221017-WA0078.opus (file attached)
	20/10/2022, 10:37 - Juan Pedro: IMG-20221020-WA0013.jpg (file attached)
	14/11/2022, 18:58 - Juan Pedro: VID-20221114-WA0039.mp4 (file attached)
	24/11/2022, 19:13 - Jaime Conde: STK-20220717-WA0090.webp (file attached)
\end{lstlisting}

En el fichero de texto plano con contenido multimedia, observaremos que el cuerpo incluye el nombre del fichero que se ha exportado con la extensión del formato del mismo. Por ello, categorizamos como \textit{voice\_message}, \textit{video\_file}, \textit{sticker} o \textit{image} en función a la extensión; \textit{.opus}, \textit{.mp4}, \textit{.webp} o \textit{.jpg}, respectivamente. Usamos estos valores puesto que son los que Telegram utiliza para su formato de mensajes.

Si el fichero no incluye estos metadatos, cada vez que un mensaje sea contenido multimedia, aparecerá \textit{\textless Media omitted \textgreater} (multimedia omitido). Estos pueden ser fotos, vídeos, música, notas de voz o documentos. Se definirán como \textit{undefined} o indefinidos, ignorándose en las visualizaciones y módulos posteriores. Se muestra un ejemplo a continuación:

\begin{lstlisting}
	17/07/2022, 01:33 - Juan Pedro: <Media omitted>
\end{lstlisting}

\subsection{Agregador}

Los mensajes se encuentran segregados en una lista, por lo que a continuación, el módulo de agregador se encargará de agregar los mensajes en diferentes grupos. En el código los hemos llamado polarizadores. Se describen los distintos submódulos a continuación:

\paragraph{Agregador por contacto}\mbox{}\\

\textit{ChatStats} se encarga de calcular las estadísticas de cada contacto para visualizarlas y mostrarlas en comparación con el resto de contactos. Hablamos de numerosos contactos, puesto que es compatible con chats individuales y grupales.

El resultado de este submódulo será un objeto \acrshort{json} con una clave por cada contacto (su nombre), que contendrá un array de los mensajes enviados por este. Se indica un ejemplo:

\begin{lstlisting}[language=JavaScript]
	{
		"Jaime": [...messagesByJaime],
		"Juan Pedro": [...messagesByJuanPedro],
		...
	}
\end{lstlisting}

donde los array de mensajes contienen objetos con la estructura de datos común mencionada en el módulo anterior,  \autoref{chap:architecture_ingesta}.

\subsubsection{Agregador por día}

Este agregador toma como entrada la salida del submódulo anterior: los mensajes agregados por contacto. Con ello se procede a agregarlos, además, por día de la semana: de lunes a domingo. Se usará el nombre del día de la semana como clave anidada.

El resultado son objetos con la siguiente estructura:

\begin{lstlisting}[language=JavaScript]
	{
		"Jaime": {
			"monday": [...messagesByJaimeOnMonday],
			"tuesday": [...messagesByJaimeOnTuesday],
			...,
			"sunday": [...messagesByJaimeOnSunday]
			},
		"Juan Pedro": {
			"monday": [...messagesByJuanPedroOnMonday],
			"tuesday": [...messagesByJuanPedroOnTuesday],
			...,
			"sunday": [...messagesByJuanPedroOnSunday]
		},
		...
	}
\end{lstlisting}

El objetivo de esta estructura de datos es visualizar la distribución de los mensajes a lo largo de la semana, en media.

\begin{comment}
	Se deja para futuras líneas un gráfico en el que se pueda elegir el año a analizar, o un scroll vertical por años.
\end{comment}

\subsubsection{Agregador por hora}

Este agregador toma también como entrada los mensajes agregados por contacto. Con ello se procede a agregarlos, además, por hora del día, usando la hora en formato 24 horas como clave anidada de agregación: de 00 a 23 horas.

El resultado son objetos con la siguiente estructura:

\begin{lstlisting}[language=JavaScript]
	{
		"Jaime": {
			"00": [...messagesByJaimeAt00],
			"01": [...messagesByJaimeAt01],
			...,
			"23": [...messagesByJaimeAt23]
		},
		"Juan Pedro": {
			"00": [...messagesByJuanPedroAt00],
			"01": [...messagesByJuanPedroAt01],
			...,
			"23": [...messagesByJuanPedroAt23]
		},
		...
	}
\end{lstlisting}

El objetivo de esta estructura de datos es visualizar la distribución de los mensajes a lo largo del día, en media.

\subsubsection{Agregador por mes}

Este agregador toma también como entrada los mensajes agregados por contacto. Con ello se procede a agregarlos, además, por MM/YYYY, por lo que deja de tratarse de un agregador acotado: pueden haber tantas claves anidadas como meses se haya hablado.

El resultado son objetos con la siguiente estructura:

\begin{lstlisting}[language=JavaScript]
	{
		"Jaime": {
			"10/2022": [...messagesByJaimeOnOctober2022],
			"11/2022": [...messagesByJaimeOnNovember2022],
			...
		},
		"Juan Pedro": {
			"10/2022": [...messagesByJuanPedroOnOctober2022],
			"11/2022": [...messagesByJuanPedroOnNovember2022],
			...
		},
		...
	}
\end{lstlisting}

El objetivo de esta estructura de datos es visualizar la distribución de los mensajes a lo largo del tiempo, con una agregación mensual.

\subsubsection{Agregador por conversación}

Este submódulo analiza las diferencias de tiempos entre los mensajes, categorizándolos como inicio de conversación si son el primer mensaje en las últimas 5 horas, o como respuesta si se trata de un intervalo de tiempo menor. Esto nos permitirá calcular cuántas conversaciones ha iniciado cada contacto, así como el tiempo de respuesta medio de cada uno; módulos que se verán más adelante.

Aunque un sesgo temporal no es una solución perfecta, acierta gran parte de las veces sin necesitar gran capacidad de cómputo.

\subsection{Cálculo de métricas}

Este módulo prepara los datos para ser representados por la librería de visualización elegida: \textit{ChartJS}.  Esta librería ha sido elegida por su alta actividad de contribuciones al proyecto, que es libre y cuenta con 12 mil estrellas en GitHub. Además, permite alta extensibilidad mediante plugins, de los que hacemos uso, por ejemplo, para mostrar las etiquetas de los datos.

Cada módulo aquí desarrollado realiza el cálculo de una métrica, cuyo resultado pasa por una función que adapta el contenido a la estructura solicitada por \textit{ChartJS}. \textit{ChartJS} nos permite un único formato de entrada para los datos, permitiéndonos elegir el tipo de gráfico de forma independiente a dichos datos. Esta arquitectura nos permite añadir más módulos posteriormente; realizando el cálculo de la métrica a mostrar y pasándolo por la función que estandariza los datos para \textit{ChartJS}.

Todos los submódulos que utilizan \textit{ChartJS} adaptan el tamaño del gráfico al número de contactos que hay en el grupo de forma \textit{responsive}. También permiten la interacción con el mismo, pudiendo eliminar contactos de la representación haciendo click sobre el nombre de los mismos en la leyenda.

Además, también se procesan datos para otras librerías de visualización, como \textit{react-wordcloud}, para las nubes de palabras o \textit{word clouds} y nubes de emoticonos.

\subsubsection{Contador de mensajes}

Este submódulo cuenta el número de mensajes enviado por cada contacto. Para chats grupales, usa la librería \textit{ChartJS}, mientras que para chat individuales, únicamente se exponen los números de ambas partes directamente.

\begin{figure}[H]
	\centering
	\subfloat[\centering Chat individual]{{\includegraphics[width=6cm]{img/message_count_individual.png} }}
	\qquad
	\subfloat[\centering Chat grupal]{{\includegraphics[width=6cm]{img/message_count.png} }}
	\caption{Contador de mensajes}
	\label{fig:chap4:message_count}
\end{figure}

Se ha optado por calcular esta métrica puesto que puede ayudar a observar diferencias drásticas en la cantidad de texto que aporta cada contacto.

\subsubsection{Contador de palabras}

Este submódulo cuenta las palabras que hay en los mensajes de cada contacto y calcula la suma total de las mismas, obteniendo el número de palabras totales enviadas por cada contacto. Para chats grupales, usa la librería \textit{ChartJS}, mientras que para chat individuales, únicamente se exponen los números de ambas partes directamente.

\begin{figure}[H]
	\centering
	\subfloat[\centering Chat individual]{{\includegraphics[width=6cm]{img/word_count_individual.png} }}
	\qquad
	\subfloat[\centering Chat grupal]{{\includegraphics[width=6cm]{img/word_count.png} }}
	\caption{Contador de palabras}
	\label{fig:chap4:word_count}
\end{figure}

\subsubsection{Contador de caracteres}

Este submódulo cuenta los caracteres que hay en los mensajes de cada contacto y calcula la suma total de los mismos, obteniendo el número de caracteres totales enviados por cada contacto. Aunque suele indicar resultados similares al módulo anterior, en algunas ocasiones es distinto, por lo que se ha decidido incluir también.

\begin{figure}[H]
	\centering
	\subfloat[\centering Chat individual]{{\includegraphics[width=6cm]{img/char_count_individual.png} }}
	\qquad
	\subfloat[\centering Chat grupal]{{\includegraphics[width=6cm]{img/char_count.png} }}
	\caption{Contador de caracteres}
	\label{fig:chap4:char_count}
\end{figure}

De nuevo, para chats grupales, usa la librería \textit{ChartJS}, mientras que para chat individuales, únicamente se exponen los números de ambas partes directamente.

\subsubsection{Media de palabras por mensaje}

Este submódulo calcula y muestra el número medio de palabras por mensaje para cada contacto. Esta métrica puede ayudar, junto con el número de mensajes totales, a saber si un contacto tiende a mandar más mensajes con menor número de palabras, o menos mensajes con más palabras.

\begin{figure}[H]
	\centering
	\subfloat[\centering Chat individual]{{\includegraphics[width=6cm]{img/word_avg_individual.png} }}
	\qquad
	\subfloat[\centering Chat grupal]{{\includegraphics[width=6cm]{img/word_avg.png} }}
	\caption{Media de palabras por mensaje}
	\label{fig:chap4:word_avg}
\end{figure}

\subsubsection{Media de caracteres por mensaje}

Este submódulo calcula y muestra el número medio de caracteres por mensaje para cada contacto. Aunque suele indicar resultados similares al módulo anterior, en algunas ocasiones puede resaltar personas que tienden a usar palabras más largas.

\begin{figure}[H]
	\centering
	\subfloat[\centering Chat individual]{{\includegraphics[width=6cm]{img/char_avg_individual.png} }}
	\qquad
	\subfloat[\centering Chat grupal]{{\includegraphics[width=6cm]{img/char_avg.png} }}
	\caption{Media de caracteres por mensaje}
	\label{fig:chap4:char_avg}
\end{figure}

\subsubsection{Número de conversaciones iniciadas}

Este submódulo calcula el número de veces que cada contacto ha iniciado la conversación, con los criterios y datos obtenidos del módulo ``Agregador por conversación''. Se ha decidido calcular esta métrica puesto que es una buena forma de medir la iniciativa de una persona, así como su interés en el grupo o persona individual.

\begin{figure}[H]
	\centering
	\subfloat[\centering Chat individual]{{\includegraphics[width=6cm]{img/conversations_started_individual.png} }}
	\qquad
	\subfloat[\centering Chat grupal]{{\includegraphics[width=6cm]{img/conversations_started.png} }}
	\caption{Conversaciones iniciadas por cada contacto}
	\label{fig:chap4:conversations_started}
\end{figure}

En este ejemplo, podemos ver cómo Alejandro, el presidente de la asociación, suele tomar la iniciativa y comenzar las conversaciones.


\subsubsection{Velocidad media de respuesta}

Este submódulo calcula la velocidad media de respuesta de cada contacto, con los criterios y datos obtenidos del módulo ``Agregador por conversación''. Se ha decidido calcular esta métrica puesto que es una buena forma de medir la atención a la aplicación de mensajería instantánea, así como la importancia que le da al grupo.

\begin{figure}[H]
	\centering
	\subfloat[\centering Chat individual]{{\includegraphics[width=6cm]{img/avg_response_time_individual.png} }}
	\qquad
	\subfloat[\centering Chat grupal]{{\includegraphics[width=6cm]{img/avg_response_time.png} }}
	\caption{Tiempo medio de respuesta}
	\label{fig:chap4:avg_response_time}
\end{figure}

Añadiendo esta figura a la anterior, podemos ver cómo Martín inicia pocas conversaciones y, además, suele tardar bastante en responder. Esto puede sugerir que está menos involucrado en el grupo.


\subsubsection{Contador de multimedia}

En caso de que existan objetos \acrshort{json} con el campo ``\textit{media\_type}'' distinto de \textit{undefined}, este submódulo cuenta cuántos archivos multimedia de cada tipo ha mandado cada contacto.


\begin{figure}[h]
	\centering
	\includegraphics[width=0.8\textwidth]{img/media_count_individual.png}
	\caption{Contador de multimedia para chats individuales}
	\label{fig:chap4:media_count_individual}
\end{figure}


\begin{figure}[h]
	\centering
	\includegraphics[width=0.7\textwidth]{img/media_count.png}
	\caption{Contador de multimedia para chats grupales}
	\label{fig:chap4:media_count}
\end{figure}

\subsubsection{Generador de estructuras de datos por día, hora y mes}

Para el gráfico de barras con el número de mensajes en el tiempo, \textit{ChartJS} necesita una estructura de datos para mensajes por día, siendo los días la variable independiente y el número de mensajes la variable dependiente.

\begin{figure}[H]
	\centering
	\includegraphics[width=0.7\textwidth]{img/time_distributions.png}
	\caption{Distribuciones en el tiempo}
	\label{fig:chap4:time_distributions}
\end{figure}

Se elige calcular y mostrar estas distribuciones puesto que ayudan a observar la evolución en el tiempo de la cantidad de mensajes intercambiados, así como los días de la semana más activos. Por ejemplo una pareja que pasa los fines de semana juntos tendrá menos mensajes los fines de semana. Por último, la distribución en las 24 horas del día ayuda a analizar las horas pico de conversación, así como las horas de final e inicio del día para cada contacto.

Para la distribución por horas en el día, se ha planteado también usar un gráfico de araña o radar, pero dicha opción se descartó al observar el solapamiento que tenía lugar con varios contactos.

En la figura podemos observar cómo el contacto Jaime comienza a mandar mensajes a las 9 de la mañana, mientras que James suele comenzar 2 horas antes: a las 7 de la mañana. Esto sugiere que James comienza antes el día, o Jaime no utiliza el móvil hasta las 9 de la mañana.

\subsubsection{Contador de palabras más repetidas}

Este submódulo procesa todas las palabras de los mensajes, elimina las palabras más comunes del español y el inglés, así como otros mensajes que WhatsApp añade, como \textit{Media ommited} o \textit{This message has been deleted}.

Las listas de palabras más comunes del español e inglés se han recopilado de distintas fuentes, combinado y eliminado repeticiones.

A la salida de este módulo, se muestra un diccionario con las palabras más repetidas y el número de veces que aparece cada una; información de la que hará uso la librería \textit{react-wordcloud}. Esta información resulta útil para resaltar los temas principales tratados en el chat.

Puede observarse en la \autoref{fig:chap4:word_emoji_cloud}

\subsubsection{Contador de emoticonos más usados}

Este módulo aplica una expresión regular unicode a todos los mensajes, seleccionando los emoticonos y contando el número de veces que aparecen. Posteriormente otra nube de palabras hace uso de esta información por medio de la librería \textit{react-wordcloud}.

\begin{figure}[H]
	\centering
	\subfloat[\centering Nube de palabras]{{\includegraphics[width=6cm]{img/word_cloud.png} }}
	\qquad
	\subfloat[\centering Nube de emoticonos]{{\includegraphics[width=6cm]{img/emoji_cloud.png} }}
	\caption{Nube de palabras y emoticonos}
	\label{fig:chap4:word_emoji_cloud}
\end{figure}

Se elige calcular y mostrar esta visualización puesto que los emoticonos constituyen en un chat la forma más similar a la expresión no verbal.

\subsection{Exportador}

Este módulo permite al usuario exportar tanto la visualización a modo de captura de pantalla, así como la exportación de las métricas calculadas para su uso personal.

\subsubsection{Descargar visualizaciones}

Descargar visualizaciones es un submódulo que permite al usuario descargar como imagen, en formato \textit{png}, las visualizaciones que se muestran en su pantalla.

Como el cliente guarda el estado de las visualizaciones en su \acrshort{dom}, haciendo uso de la librería \textit{html2canvas}, podemos acceder al objeto \textit{body} de la página y obtener una imagen canvas del mismo. Esta imagen no es una captura de pantalla, sino una reconstrucción del objeto \textit{body} del \acrshort{dom}, por lo que puede no ser completamente fiel a la representación que el usuario tiene en su cliente.

Esta imagen se codifica en base64 y se crea un enlace con tipo de contenido (\textit{Content-Type}) \textit{image/png}  para su descarga local desde el navegador.

No se puede hacer uso de la funcionalidad de imprimir pantalla del navegador, puesto que los gráficos usan la etiqueta \acrshort{html} \textit{canvas}, lo que descuadra los gráficos y hace perder formato y colores.

\subsubsection{Descargar resultados}

Del mismo modo que el submódulo anterior, este módulo permite al usuario descargar una versión exportada de los datos y métricas calculadas en formato \acrshort{json} para su posterior uso personal.

Para ello, el submódulo recoge todo el contenido alojado en el estado local del cliente y crea un enlace local para la descarga de los datos. Dicho enlace está compuesto por el tipo de contenido (\textit{Content-Type}) \textit{text/json} seguido del contenido del estado local (en \acrshort{json}) volcado a una cadena de caracteres y codificado para \acrshort{uri}s.

\todi{Enlazar el apartado con la lista de métricas que se exportan (más arriba)}



\section{Conclusión}

Con todo, si en un futuro se quiere extender la arquitectura propuesta a un modelo de cliente-servidor, siempre se puede añadir un tercer contenedor en la última capa de la \autoref{chap:architecture:server} que implemente lógica adicional. Sería también necesaria cierta modificación en el código del cliente para implementar peticiones a este nuevo servicio.


\chapter{Caso de estudio}
\label{chap:case-study}

\section{Introducción}
\label{sec:introduction}
In this chapter we are going to describe a selected use case. This description will cover the main Wool features, and its main purpose is to completely understand the functionalities of Wool, and how to use it. 
\section{Rule edition}
...
\chapter{Conclusiones y futuras líneas de trabajo}
\label{chap:conclusions}

En este capítulo se exponen las conclusiones extraídas del proyecto, así como recomendaciones y posibles futuras líneas de trabajo.

\section{Conclusiones}
\label{sec:conclusions}

Académicamente, este proyecto ha desarrollado una doble función: la de proyecto personal y, la de trabajo de fin de grado.

La motivación principal del proyecto era desarrollar una aplicación web para ayudar a analizar conversaciones de WhatsApp y Telegram, permitiendo detectar problemas y tomar acciones para mejorar. Personalmente, este proyecto ha servido para mejorar amistades y relaciones de pareja, así como ayudado a comprender mejor el comportamiento de mis grupos de amigos y asociaciones de estudiantes.

\begin{comment}
	
Academically, this Project served a double purpose: learning about statistics and machine learning and developing a method for improving the accuracy of an existing system.

The initial motivation for the Project was to develop better estimations for the bus \acrshort{eta}s, and this main objective has been satisfactorily accomplished. Precisely, both of the goals proposed in the introduction (characterizing the \acrshort{crtm} \acrshort{api} estimations and developing alternative estimators) have been fullfilled.

The first step was to review the existing bibliography and check the available data and data sources from the \acrshort{crtm}. Then, we addressed the first challenge: obtaining as much data about the buses mobility as possible without saturating the \acrshort{crtm} \acrshort{api} server. After several tests, the server behaviour was characterized and decisions for an optimal data gathering were taken. Furthermore, the Python package \textbf{crtm\_poll} was developed for automating the server polling process. As a conclusion, the optimization of the data acquisition procedures is crucial in this type of scenarios.

Once the data was available, a data analysis environment was set up for its processing. An algorithm for estimating the bus passing time from the \acrshort{api} provided \acrshort{eta}s was designed, which involved filtering the raw data obtained from the \acrshort{crtm} \acrshort{api}. Knowing the arrival times of past trips, the running time between stops could be obtained and a web visualization tool was developed for displaying them. The developed tools allowed to visually detect some fundamental behavior patterns in the mobility of the buses.

Next, alternative estimators for the running times between stops and for the bus \acrshort{eta} to a stop were designed. For that purpose different statistical tools and machine learning models were employed. Our experience indicates that although the available software for machine learning model design is very powerful, a clear understanding of the addressed problem and of the training schemes is required to be able to correctly interpret the results.

Concerning the developed tools, the running time between stops estimators developed in Section \ref{running_time_between_two_stops} could be used to improve the accuracy of the estimations made by the \acrshort{crtm} \acrshort{api} or by other source of estimators; in general, they can also be applied to any scenario where we want to estimate the running time between two points of a vehicle following a fixed route.

The relevant idea behind the estimator developed in Section \ref{reamining_time_at_anytime} is that it can be understood as a system that corrects the estimations provided by another system. This could be an easy to implement solution for improving the accuracy of the estimations made by more complex systems just by setting it prior to them.

The main limitation for implementing the proposed online estimator is the bottleneck produced by the \acrshort{crtm} \acrshort{api} server, which would not be able to handle all the needed requests for sampling all the bus stops at the same time. Nevertheless, the computational cost of each of the used models was calculated in case that the processing capacity requirements were to be consider for a future implementation.

In general, the developed methods in this Project are built in a way that allows the usability for other similar use cases. The hyperparameters of the estimators are chosen based on the data and the available input features, so they can be easily adapted for different system behaviors.
\end{comment}

\section{Objetivos conseguidos}
\label{sec:achieved-goals}

Finalmente, hemos logrado alcanzar todos los objetivos propuestos en el \autoref{chap:introduction}:

\begin{itemize}
	
	\item Importar datos conversacionales de aplicaciónes de mensajería instantánea.
	\item Selección y cálculo de estadísticas de conversaciones.
	\item Diseño y desarrollo de las visualizaciones.
	\item Integración multiplataforma.
	\item Aplicación de técnicas de ingeniería de software.
	
\end{itemize}

\section{Futuras líneas de trabajo}
\label{sec:future-work}

\begin{itemize}

\item Migración de los módulos con mayor carga de trabajo a \acrfull{wasm}, permitiendo obtener un rendimiento mayor y cercano al nativo del cliente que ejecuta la aplicación.

\item Mayor facilidad para añadir módulos, mediante un sistema de carpetas y plugins.

\item Inclusión de visualizaciones para el análisis de sentimientos y su evolución en el tiempo.

\item Inclusión de mensajes eliminados por cada contacto.

\item Detección de grupos de interacción: qué contactos suelen responder entre ellos.

\item Implementación de una \textit{cache} en la \acrfull{pwa} para poder ejecutar la aplicación sin necesidad de acceso a Internet (una vez instalada). Actualmente, la aplicación puede instalarse, pero cada vez que se abre solicita el código al servidor.

\item Mejora de la documentación para futuros contribuyentes al proyecto.

\end{itemize}


\cleardoublepage
\pagenumbering{roman}
\appendix
\begin{appendices}
    \chapter{Impacto del proyecto} \label{chap:impact}

\section{Impacto social} \label{sec:social_impact}

\subsection{Introducción}

Este proyecto aporta herramientas para el estudio y análisis de las relaciones personales llevadas a cabo mediante las aplicaciones WhatsAp y Telegramp, permitiendo evaluar y mejorar y tomar decisiones las mismas mediante la observación de los resultados.

\subsection{Descripción de impactos relevantes relacionados con el proyecto}

El cuidado de las comunicaciones que tienen lugar a través de WhatsApp puede ayudar a las personas a dedicar el tiempo necesario a mantener relaciones saludables y darse cuenta de posibles fallos que están teniendo lugar en la comunicación a largo plazo, o analizar su comportamiento.

Otro impacto relevante a la ética del proyecto es las consideraciones de privacidad en un tema tan importante como son las conversaciones privadas del usuario, que se ha considerado prioritario durante las decisiones de diseño y arquitectura.

\subsection{Conclusiones}

Las consideraciones sociales de este proyecto son el núcleo del desarrollo del mismo, así como la causa por la que el proyecto comenzó en un primer lugar.
    \chapter{Presupuesto económico} \label{chap:economic}

\section{Costes}

\ \hfill
\begin{tabular}{ p{5.5cm}|p{2.7cm}|p{2.7cm}|p{2cm}|  }
	\cline{2-4}
	& \multicolumn{3}{|c|}{\textbf{COSTE DE MANO DE OBRA (coste directo)}} \\
	\cline{2-4}
	& Horas & Precio/hora & Total \\
	\cline{2-4}
	& 360 & \hfill 12 \euro & \hfill 4,320.00 \euro \\
	\cline{2-4}
\end{tabular}

\vspace{1cm}

\ \hfill
\begin{tabular}{ |p{3cm}| *{4}{p{2.1cm}|}  }
	\hline
	\multicolumn{5}{|c|}{\textbf{COSTE DE RECURSOS MATERIALES (coste directo)}} \\
	\hline
	& Precio de compra & Uso en meses & Amortización (en años) & Total \\
	\hline
	Portátil personal & \hfill 1,500.00 \euro & 6 & 5 & \hfill 150.00 \euro \\
	\hline
	Tablet & \hfill 1,099.00 \euro & 6 & 5 & \hfill 109.90 \euro \\
	\hline
	Escritorio ETSIT & \hfill 130 \euro & 6 & 5 & \hfill 15.60 \euro \\
	\hline
	Silla ETSIT & \hfill 50 \euro & 6 & 5 & \hfill 5.00 \euro \\
	\hline
	\multicolumn{4}{|l|}{COSTE TOTAL DE RECURSOS MATERIALES} & \hfill 280.50 \euro \\
	\hline
\end{tabular}

\vspace{1cm}

\ \hfill
\begin{tabular}{ |p{8.4cm}|p{0.8cm}|p{2.1cm}|p{2cm}| }
	\hline
	\textbf{GASTOS GENERALES (costes indirectos)} & 15\% & de CD & \hfill 690.08 \euro \\
	\hline
	\textbf{BENEFICIO INDUSTRIAL} & 6\% & of CD+CI & \hfill 317.43 \euro \\
	\hline
\end{tabular}

\vspace{1cm}

\ \hfill
\begin{tabular}{ |p{5cm}|p{2cm}|  }
	\hline
	\multicolumn{2}{|c|}{\textbf{MATERIAL FUNGIBLE}} \\
	\hline
	Impresión y encuadernación & \hfill 50.00 \euro \\
	\hline
\end{tabular}

\vspace{1cm}

\ \hfill
\begin{tabular}{ |p{4cm}|p{2cm}|p{2cm}|  }
	\hline
	\multicolumn{2}{|l|}{\textbf{PRESUPUESTO SUBTOTAL}} & \hfill 5,658.01 \euro \\
	\hline
	\textbf{IVA APLICABLE} & 21\% & \hfill 1,188.18 \euro \\
	\hline
\end{tabular}

\vspace{1cm}

\ \hfill
\begin{tabular}{ |p{5cm}|p{2cm}|}
	\hline
	\textbf{PRESUPUESTO TOTAL} & \hfill 6,846.19 \euro \\
	\hline
\end{tabular}

    \chapter{Código} \label{chap:code}

\section{Acceso al código fuente}

Acceso principal: \url{https://codeberg.org/devve/chatstats}

Copia (mirror): \url{https://github.com/d3vv3/chatstats}


    \chapter{Expresiones regulares} \label{chap:regex}

\section{Grupos de captura}

Como se decribe en el \autoref{chap:architecture}, poder separar cada mensaje se ha llegado a la siguiente expresión regular, cuyos grupos se explicaran a continuación:

\begin{lstlisting}
	/(\d{2}\/\d{2}\/\d{4}),\s(\d(?:\d)?:\d{2})\s-\s([^:]*):\s(.*?)(?=\s*\d{2}\/\d{2}\/\d{4},\s|$)/ug
\end{lstlisting}

Se denotan los distintos grupos de captura por las agrupaciones realizadas con los paréntesis. Se exponen:

\paragraph{Grupo de captura 1: fecha}\mbox{}\\

\begin{lstlisting}
	(\d{2}\/\d{2}\/\d{4})
\end{lstlisting}

Se encarga de la fecha en formato \textit{dd/mm/YYYY}, denotado con ``$\backslash d\{X\}$'' que indica que se buscan $X$ dígitos de $0$ a $9$ seguidos, separados por un ``$/$''. En las expresiones regulares hay que escapar los ``$/$'' o \textit{slash} con un ``$\backslash$'' o \textit{backslash}.

Durante numerosas pruebas se ha observado que la fecha siempre sigue este formato, independientemente del sistema operativo. No hay consistencia entre los ajustes del parámetro \textit{locale} de \textit{en\_US} y \textit{es\_ES}, siendo \textit{mm/dd/YYY} y \textit{dd/mm/YYYY} respectivamente. Para ello, el cliente accederá al \textit{locale} para actuar en consecuencia más adelante. No se ha probado para otras configuraciones.

\paragraph{Grupo de captura 2: hora}\mbox{}\\

\begin{lstlisting}
	(\d(?:\d)?:\d{2})
\end{lstlisting}

Se encarga de la la hora en formato \textit{hh:MM}, aunque en alguna ocasión se ha observado que no hay consistencia si la hora es de un solo dígito, pudiendo aparecer 01:00 o 1:00 en función del dispositivo y la versión de WhatsApp que ejecuta. Es por ello que la expresión regular es algo más compleja y se buscan dígitos a la izquierda del carácter ``\textit{:}'' independientemente del número de repeticiones del mismo. Los minutos si han mostrado consistencia, por lo que se utiliza ``$\backslash d\{2\}$''.

\paragraph{Grupo de captura 3: contacto}\mbox{}\\

\begin{lstlisting}
	([^:]*)
\end{lstlisting}

Se encarga del nombre del contacto. Busca la repetición de caracteres ilimitados a excepción del carácter ``\textit{:}'', ya que éste es un separador.

\paragraph{Grupo de captura 4: mensaje}\mbox{}\\

\begin{lstlisting}
	(.*?)
\end{lstlisting}

Se encarga del cuerpo del mensaje. Busca la repetición de caracteres ilimitados, incluyendo caracteres unicode para tener los emoticonos en cuenta. Esta búsqueda de caracteres se realiza de manera perezosa, expandiendo las coincidencias en caso posible, siempre que no coincida con el siguiente grupo de captura (\textit{lookahead}).

\paragraph{Look ahead o mirada hacia delante}\mbox{}\\

\begin{lstlisting}
	(?=\s*\d{2}\/\d{2}\/\d{4},\s|$)
\end{lstlisting}

Si únicamente contáramos con el grupo de captura 4, solo se reconocería el primer mensaje, puesto que se reconocería el resto del texto como cuerpo del primer mensaje. Para solucionarlo, en el grupo de captura 4 se intentan reconocer el menor número posible de coincidencias, hasta el siguiente patrón reconocido. Este patrón es una mirada hacia delante conformada por la misma expresión regular que en el grupo de captura 1, acompañada seguida de una coma ``\textit{,}'' y un espacio.
\end{appendices}

\cleardoublepage
\phantomsection
\nocite{*}
\addcontentsline{toc}{chapter}{Bibliography}   % ieeetr
\bibliographystyle{unsrt}
\bibliographystyle{plain}{
    \small
    \bibliography{biblio/ref}
}

\end{document}