\cleardoublepage
\phantomsection
\chapter*{Resumen}
\addcontentsline{toc}{chapter}{Resumen}

Con 14 años compré mi primer teléfono inteligente. Con ello, gran parte de las interacciones con las personas de mi entorno más cercano, tenían lugar a través de aplicaciones de mensajería instantánea: principalmente WhatsApp. Por entonces no era consciente de la información que se perdía por estas vías, pero con el tiempo me he dado cuenta de la importancia de la información no verbal que se manifiesta en gestos, emociones, entonación y; finalmente, con la acumulación de todos estos factores, puede llegar a deteriorar una relación si no se reacciona correctamente.

Han sido numerosas las ocasiones en las que he sentido la necesidad de comprobar si mi relación con alguna persona cercana estaba decayendo, o se trataba únicamente de un sentimiento sin fundamentos. Es por ello que, en 2019, comencé el desarrollo de una herramienta de código libre para el análisis de conversaciones exportadas por aplicaciones de mensajería instantánea. Al tratar información tan personal, la arquitectura estaba clara: todo el procesamiento debe hacerse en el cliente y evitar enviar información a servidores salvo la autorización del usuario final.

\vfill
\textbf{Palabras clave: mensajería instantánea, aplicación web, javascript, código libre, WhatsApp, privacidad} 